\input{../../template.ltx}
\usepackage{multirow}

\begin{document}

\osuetitle{3}

\section*{Aufgabenstellung -- mrna}

In diesem Beispiel soll einer der grundlegendsten Vorgänge in der Zelle, die
Synthese von Proteinen (auch Translation genannt), anschaulich dargestellt
werden.

Die \emph{mRNA} (messenger RNA), auch Boten-RNA genannt, entspricht einem
Teilabschnitt der DNA und wird aus dieser synthetisiert (auch Transkription
genannt).  Das daraus resultierende Template dient anschließend in der Zelle
zur Produktion von unterschiedlichsten \emph{Proteinen}.

Ein Protein besteht aus der Kombination von 20 verschiedenen
\emph{Aminosäuren} und die Information, welche Aminosäuren bei der Synthese
aneinander gehängt werden müssen, werden in der mRNA gespeichert. Den dazu
verwendeten Code nennt man auch den \emph{genetischen Code}. (Siehe
\textit{Genetic Code})

Die mRNA verwendet keinen binären Code, jedoch erinnert der Aufbau sehr an
Konzepte aus der Informatik. Statt Bits [\verb|0| oder \verb|1|; Basis 2]
werden \emph{Nukleinbasen} [Adenin (\verb|A|), Guanin (\verb|G|), Cytosin
(\verb|C|) oder Uracil (\verb|U|); Basis 4] zur Speicherung von Informationen
verwendet. 8 Bits werden immer zu einem Byte zusammengefasst, 3 Nukleinbasen zu
einem \emph{Codon}. Dieses Codon kann nun einem der 20 Aminosäuren zugeordnet
werden. Vergleichbar: ein Byte entspricht einem ASCII Zeichen in der
ASCII-Tabelle. Da es aber nur 20 Aminosäuren gibt, jedoch $4^3=64$ mögliche
Codons, sind einige Aminosäuren mehrfach belegt.
Tabelle~\ref{tab:geneticcode} zeigt die Zuordnung der Nukleinbasentripel
zu den Aminosäuren.

Beispiel: UUU=Phe(F), UUC=Phe(F), AAA=Lys(K), AAG=Lys(K)

Eine Protein-Sequenz (die message) innerhalb einer mRNA-Sequenz beginnt immer
mit einem \emph{Start-Codon} (AUG) und endet mit einem \emph{Stopp-Codon}
(UAA, UAG, UGA). Die dazwischen liegenden Codons geben an, aus welchen
Aminosäuren das Protein schlussendlich synthetisiert werden soll. Ein
Beispiel illustriert den Vorgang:

\renewcommand{\arraystretch}{1.4}
\begin{center}
\begin{tabular}{r c}
\textit{mRNA} & … CACAGCCU AUG GUA GGG UCU GCU UAA CGCCGC ... \\
 & AUG (START) | GUA=\textbf{V}, GGG=\textbf{G}, UCU=\textbf{S}, GCU=\textbf{A} | UAA (STOP) \\
\textit{Protein} & \textbf{VGSA}
\end{tabular}
\end{center}


\renewcommand{\arraystretch}{1.4}
\begin{table}
\centering
\begin{tabular}{|c|c|c|c|c|c|c|c|c|c|c|c|c|c|c|c|c|c|c|}
\cline{1-4} \cline{6-9} \cline{11-14} \cline{16-19}
\multirow{16}{*}{U} & \multirow{4}{*}{U} & U & \multirow{2}{*}{\textbf{\begin{tabular}[c]{@{}c@{}}Phe\\ F\end{tabular}}} & & \multirow{16}{*}{C} & \multirow{4}{*}{U} & U & \multirow{4}{*}{\textbf{\begin{tabular}[c]{@{}c@{}}Leu\\ L\end{tabular}}} & & \multirow{16}{*}{A} & \multirow{4}{*}{U} & U & \multirow{3}{*}{\textbf{\begin{tabular}[c]{@{}c@{}}Ile\\ I\end{tabular}}} & & \multirow{16}{*}{G} & \multirow{4}{*}{U} & U & \multirow{4}{*}{\textbf{\begin{tabular}[c]{@{}c@{}}Val\\ V\end{tabular}}} \\ \cline{3-3} \cline{8-8} \cline{13-13} \cline{18-18}
& & C & & & & & C & & & & & C & & & & & C & \\ \cline{3-4} \cline{8-8} \cline{13-13} \cline{18-18}
& & A & \multirow{2}{*}{\textbf{\begin{tabular}[c]{@{}c@{}}Leu\\ L\end{tabular}}} & & & & A & & & & & A & & & & & A & \\ \cline{3-3} \cline{8-8} \cline{13-14} \cline{18-18}
& & G & & & & & G & & & & & G & \textbf{\begin{tabular}[c]{@{}c@{}}Met M\\START\end{tabular}} & & & & G & \\ \cline{2-4} \cline{7-9} \cline{12-14} \cline{17-19}
& \multirow{4}{*}{C} & U & \multirow{4}{*}{\textbf{\begin{tabular}[c]{@{}c@{}}Ser\\ S\end{tabular}}} & & & \multirow{4}{*}{C} & U & \multirow{4}{*}{\textbf{\begin{tabular}[c]{@{}c@{}}Pro\\ P\end{tabular}}} & & & \multirow{4}{*}{C} & U & \multirow{4}{*}{\textbf{\begin{tabular}[c]{@{}c@{}}Thr\\ T\end{tabular}}} & & & \multirow{4}{*}{C} & U & \multirow{4}{*}{\textbf{\begin{tabular}[c]{@{}c@{}}Ala\\ A\end{tabular}}} \\ \cline{3-3} \cline{8-8} \cline{13-13} \cline{18-18}
& & C & & & & & C & & & & & C & & & & & C & \\ \cline{3-3} \cline{8-8} \cline{13-13} \cline{18-18}
& & A & & & & & A & & & & & A & & & & & A & \\ \cline{3-3} \cline{8-8} \cline{13-13} \cline{18-18}
& & G & & & & & G & & & & & G & & & & & G & \\ \cline{2-4} \cline{7-9} \cline{12-14} \cline{17-19}
& \multirow{4}{*}{A} & U & \multirow{2}{*}{\textbf{\begin{tabular}[c]{@{}c@{}}Tyr\\ Y\end{tabular}}} & & & \multirow{4}{*}{A} & U & \multirow{2}{*}{\textbf{\begin{tabular}[c]{@{}c@{}}His\\ H\end{tabular}}} & & & \multirow{4}{*}{A} & U & \multirow{2}{*}{\textbf{\begin{tabular}[c]{@{}c@{}}Asn\\ N\end{tabular}}} & & & \multirow{4}{*}{A} & U & \multirow{2}{*}{\textbf{\begin{tabular}[c]{@{}c@{}}Asp\\ D\end{tabular}}} \\ \cline{3-3} \cline{8-8} \cline{13-13} \cline{18-18}
& & C & & & & & C & & & & & C & & & & & C & \\ \cline{3-4} \cline{8-9} \cline{13-14} \cline{18-19}
& & A & \multirow{2}{*}{\textbf{STOP}} & & & & A & \multirow{2}{*}{\textbf{\begin{tabular}[c]{@{}c@{}}Gln\\ Q\end{tabular}}} & & & & A & \multirow{2}{*}{\textbf{\begin{tabular}[c]{@{}c@{}}Lys\\ K\end{tabular}}} & & & & A & \multirow{2}{*}{\textbf{\begin{tabular}[c]{@{}c@{}}Glu\\ E\end{tabular}}} \\ \cline{3-3} \cline{8-8} \cline{13-13} \cline{18-18}
& & G & & & & & G & & & & & G & & & & & G & \\ \cline{2-4} \cline{7-9} \cline{12-14} \cline{17-19}
& \multirow{4}{*}{G} & U & \multirow{2}{*}{\textbf{\begin{tabular}[c]{@{}c@{}}Cys\\ C\end{tabular}}} & & & \multirow{4}{*}{G} & U & \multirow{4}{*}{\textbf{\begin{tabular}[c]{@{}c@{}}Arg\\ R\end{tabular}}} & & & \multirow{4}{*}{G} & U & \multirow{2}{*}{\textbf{\begin{tabular}[c]{@{}c@{}}Ser\\ S\end{tabular}}} & & & \multirow{4}{*}{G} & U & \multirow{4}{*}{\textbf{\begin{tabular}[c]{@{}c@{}}Gly\\ G\end{tabular}}} \\ \cline{3-3} \cline{8-8} \cline{13-13} \cline{18-18}
& & C & & & & & C & & & & & C & & & & & C & \\ \cline{3-4} \cline{8-8} \cline{13-14} \cline{18-18}
& & A & \textbf{STOP} & & & & A & & & & & A & \multirow{2}{*}{\textbf{\begin{tabular}[c]{@{}c@{}}Arg\\ R\end{tabular}}} & & & & A & \\ \cline{3-4} \cline{8-8} \cline{13-13} \cline{18-18}
& & G & \textbf{\begin{tabular}[c]{@{}c@{}}Trp\\ W\end{tabular}} & & & & G & & & & & G & & & & & G & \\ \cline{1-4} \cline{6-9} \cline{11-14} \cline{16-19}
\end{tabular}
  \caption{Genetischer Code: Dekodierung der Codons zu Aminosäuren}
  \label{tab:geneticcode}
\end{table}


\subsection*{Anleitung}

Die Implementierung besteht aus einem Server, der die mRNA Sequenz und Befehle
entgegennimmt und entsprechend verarbeitet, und einem Client, der zur Eingabe
von Daten dient und die Ergebnisse visualisiert. Wählen Sie eine geeignete
maximale Länge der mRNA- und Protein-Sequenz. Die Kommunikation zwischen den
Prozessen soll mittels Shared Memory realisiert werden und die Synchronisierung
über Semaphore erfolgen.

Die Kommunikation zwischen den Clients und dem Server soll mittels einem
einzigen Shared Memory Object erfolgen (\textbf{nicht} einem pro Client),
welches nur die Information, die der Server mit einem einzigen Client
austauscht enthalten darf.  Allerdings muss eine beliebige Anzahl von Clients
gleichzeitig und unabhängig voneinander mit dem Server kommunizieren können.
Insbesondere darf das Warten auf den Input eines Clients nicht andere Clients
blockieren.

Server und Client sollen die Freigabe des Shared Memory Objects und der
Semaphore koordinieren. Spätestens wenn der Server und alle Clients terminiert
haben, müssen alle Semaphore und Shared Memory Objects freigegeben sein.


\subsection*{Client}

\begin{verbatim}
USAGE: mrna-client
\end{verbatim}

Wenn der Client gestartet wird, soll dem Nutzer zuerst eine Liste aller
verfügbaren Befehle ausgegeben werden:

\begin{verbatim}
$ ./mrna-client
Available commands:
 s - submit a new mRNA sequence
 n - show next protein sequence in active mRNA sequence
 r - reset active mRNA sequence
 q - close this client
\end{verbatim}

Die Auswahl soll von der Standardeingabe gelesen werden.

\begin{description}

  \item[\texttt{s}] Dieser Befehl erlaubt es dem Nutzer eine neue mRNA-Sequenz
    an den Server zu senden. Der Server hinterlegt diese Sequenz in einem dem
    Client zugeordneten Datenspeicher. Die Eingabe wird durch zwei Leerzeilen
    beendet und es sollen alle ungültigen Zeichen und Leerraum (Tabs,
    Leerzeichen, Zeilenumbrüche) entfernt werden.

  \item[\texttt{n}] Sollte für diesen Client eine mRNA-Sequenz am Server
    hinterlegt sein, gibt dieser Befehl das nächste Vorkommen einer
    Protein-Sequenz aus. Neben der gefundenen Protein-Sequenz selbst wird
    sowohl die Start als auch die End-Position in der betreffenden mRNA-Sequenz
    ausgegeben.

  \item[\texttt{r}] Wenn das Ende einer aktiven mRNA-Sequenz erreicht wurde,
    kann der Nutzer mit diesem Befehl den Cursor für diesen Client
    zurückstellen. Dadurch kann wieder die gesammte mRNA-Sequenz mit dem
    n-Command durchlaufen werden.

  \item[\texttt{q}] Dieser Befehl schließt den Client und gibt alle Ressourcen
    frei die der Client hält und die der Server für den Client hält
    (z.B.\ die aktive mRNA-Sequenz oder die Cursor-Position).

\end{description}


\subsection*{Server}

\begin{verbatim}
USAGE: mrna-server
\end{verbatim}

Der Server legt zu Beginn die benötigten Resourcen an.
Der Server soll es erlauben, dass mehrere unabhängige Clients mit ihm
gleichzeitig kommunizieren können. Es soll sichergestellt sein, dass jeder
Client nur seine eigene mRNA-Sequenz sieht und es in keinem Fall zu Deadlocks
oder anderen Synchronisationsfehlern kommt.

Empfängt der Server einen n-Befehl, so bewegt er einen Cursor durch die aktive
mRNA-Sequenz bis er ein Start-Codon (AUG) entdeckt. Nun wird jedes folgende
Codon einer Aminosäure zugeordnet bis eines der Stopp-Codons (UAA, UAG, UGA)
erreicht wird. Die gefundene Protein-Sequenz wird dann über den Client
ausgegeben. Die aktuelle Cursor-Position bleibt am Server gespeichert.

\subsection*{Beispiel}

Folgendes Beispiel zeigt den Umfang und das zu verwendende Format der
Ausgabe des Clients.


\begin{verbatim}
$ ./mrna-client
Available commands:
 s - submit a new mRNA sequence
 n - show next protein sequence in active mRNA sequence
 r - reset active mRNA sequence
 q - close this client
Enter new command. (Enter to end input): s
Enter new mRNA sequence. (Newline to end input):
CACAGCCUAUGGUAGGGUCUGCUUAACGCCGCUAUGUAUCAUAAGGAGACAUGACGCCGC

Enter new command. (Enter to end input): n
Protein sequence found [11/60] to [23/60]: VGSA
Enter new command. (Enter to end input): n
Protein sequence found [36/60] to [51/60]: YHKET
Enter new command. (Enter to end input): n
End reached [60/60], send r to reset.
Enter new command. (Enter to end input): r
Reset. [0/60]
Enter new command. (Enter to end input): n
Protein sequence found [11/60] to [23/60]: VGSA
Enter new command. (Enter to end input): q
Close client.
\end{verbatim}




\osueguidelinesthree

\end{document}
