\input{../../template.ltx}

\begin{document}

\osuetitle{3}

\section*{Aufgabenstellung -- storetool}

Schreiben Sie eine Store-Tool-Anwendung. Diese Anwendung besteht aus zwei
Programmen.

\begin{verbatim}
    SYNOPSIS
        storeserver
        storetool  <-s key data | -g key | -r key | -q>
\end{verbatim}

Die beiden Programme kommunizieren über ein Shared-Memory-Segment. Das
Store-Tool schickt Kommandos an den Server. Danach wartet es, bis der Server das
Kommando abgearbeitet hat, überprüft das Ergebnis und terminiert.

Gespeichert werden Strings, die Sie jeweils auf 64 Zeichen inklusive
terminierendem \verb_'\0'_ (Konstante definieren!) begrenzen können. Jeder
String wird mit einem positiven, ganzzahligen Schlüssel (\osuearg{key})
assoziiert.

\subsection*{Kommandos}

\osueprog{storetool} akzeptiert bei jedem Aufruf genau eines der folgenden
Kommandos:

\begin{center}
\begin{tabular}{@{}llp{10cm}@{}}
\toprule
store  & \osuearg{-s key data} & Der Server soll die Daten (\osuearg{data})
                                 assoziiert mit \osuearg{key} speichern.
                                 Existiert \osuearg{key} bereits, meldet der
                                 Server einen Fehler ans Store-Tool.

                                 Verwenden Sie zum Speichern eine einfach
                                 verkettete Liste. \\
\midrule
get    & \osuearg{-g key}      & Das Store-Tool fordert vom Server die mit
                                 \osuearg{key} assoziierten Daten an. Wenn keine
                                 Daten gespeichert sind, soll der Server einen
                                 Fehler an das Store-Tool melden. Tritt kein
                                 Fehler auf, soll das Store-Tool die Daten auf
                                 \osueglvar{stdout} ausgeben. \\
\midrule
remove & \osuearg{-r key}      & Der Server soll einen Wert aus der Liste
                                 löschen. Existiert \osuearg{key} nicht, wird
                                 an das Store-Tool ein Fehler gemeldet. \\
\midrule
quit   & \osuearg{-q}          & Erhält der Server dieses Kommando, soll er sich
                                 beenden. \\
\bottomrule
\end{tabular}
\end{center}

\subsection*{Hinweis}

Nicht-kritische Fehlermeldungen (jene, die sich auf die Liste beziehen und vom
Server an das Store-Tool gemeldet wurden) sollen vom Store-Tool auf
\osueglvar{stderr} ausgegeben werden.

Es ist darauf zu achten, dass mehrere Store-Tools gleichzeitig gestartet werden
können. Die anderen Store-Tools müssen in diesem Fall so lange warten, bis das
erste das Ergebnis vom Server gelesen hat. Bei dem Kommando \osueinput{quit}
dürfen schon wartende Store-Tools mit einem Fehler terminieren, nicht aber der
Server und jenes Store-Tool, das das Kommando sendet.

\subsection*{Anleitung}

Schreiben Sie zwei Programme, die die Prozesse mittels einer
Client/Server-Struktur realisieren. Achten Sie auf eine saubere Terminierung.

\osueguidelinesthree

\end{document}
