% TeX source file
% Sysprog WS 2005
% Beispiel 2: caesar
% Dietmar Schabus (e0147322@student.tuwien.ac.at)
\input{../../template.ltx}

\begin{document}

\osuetitle{3}

\section*{Aufgabenstellung}

\begin{verbatim}
SYNOPSIS:
     readin <filename>
     caesar <key>
\end{verbatim} 

Implementieren Sie zwei Programme {\tt readin} und {\tt caesar}, die
einen einfachen (und leicht zu knackenden!)
Verschl"usselungsalgorithmus anwenden.

Das Programm {\tt readin} liest zeilenweise aus einem als Argument
"ubergebenen Textfile, und schreibt jede Zeile nacheinander auf
folgende Weise modifiziert in ein Shared Memory:
\begin{itemize}
\item Alle Gro"sbuchstaben (alle Zeichen in {\tt [A-Z]}) werden
unver"andert beibehalten.
\item Alle Kleinbuchstaben (alle Zeichen in {\tt [a-z]}) werden in
Gro"sbuchstaben konvertiert.
\item Alle anderen Zeichen fallen weg.
\end{itemize}
Das Programm {\tt caesar} erwartet als Argument eine Zahl zwischen 1
und 25. Es liest jede Zeile aus dem Shared Memory und ersetzt jeden
Buchstaben durch den im Alphabet um {\tt key} weiter rechts stehenden
Buchstaben (wobei sich rechts neben dem Z wieder das A befinden
soll). Bei einem {\tt key} von 3 wird also z.B. A zu D, B zu E, C zu
F, W zu Z, X zu A und Y zu B. Die so verschl"usselte Zeile wird dann
auf {\tt stdout} ausgegeben, und die n"achste Zeile wird gelesen.

{\tt caesar} soll die n"achste Zeile erst dann lesen, wenn {\tt
readin} sie vollst"andig ins Shared Memory geschrieben hat und {\tt
readin} soll erst dann die n"achste Zeile ins Shared Memory schreiben,
wenn die vorherige schon von {\tt caesar} gelesen wurde.

Sobald {\tt readin} die letzte Zeile des Eingabefiles verarbeitet hat,
schreibt es einen frei w"ahlbaren String in das Shared Memory, der
{\tt caesar} anzeigt, dass alle Arbeit getan ist. Darauf sollen beide
Programme terminieren.



\subsection*{Anleitung}
Folgende vereinfachenden Einschr"ankungen gelten:
\begin{itemize}
\item Im Eingabefile stehen nur ASCII-Zeichen. Umlaute und dergleichen
m"ussen also nicht behandelt werden.
\item Keine Zeile ist l"anger als 1022 Zeichen (inklusive Newline).
\end{itemize}




\subsection*{Testen}
Testen Sie Ihr Programm mit mehreren Eingabefiles. 
Erstellen Sie z.B. ein Testfile \emph{t1} mit folgendem Inhalt: 
\begin{verbatim}
Ich habe das dritte Sysprog-Beispiel,
das gar nicht schwer war,
schon fast fertig.
\end{verbatim}
Rufen Sie dann die beiden Programme auf (Reihenfolge je nach Ihrer
Implementierung):
\begin{verbatim}
  readin t1 &
  caesar 15 > t2
\end{verbatim}
Dann haben Sie in \emph{t2} die verschl"usselte Version von \emph{t1}.

Mit
\begin{verbatim}
  readin t2 &
  caesar 11
\end{verbatim}
k"onnen Sie diese wieder entschl"usseln, und sollten folgende Ausgabe
erhalten:
\begin{verbatim}
ICHHABEDASZWEITESYSPROGBEISPIEL
DASGARNICHTSCHWERWAR
SCHONFASTFERTIG
\end{verbatim}

\osueguidelinesthree

\end{document}
