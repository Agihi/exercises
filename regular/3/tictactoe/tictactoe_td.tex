\input{../../template.ltx}

\begin{document}

\osuetitle{3}

\section*{Aufgabenstellung -- tictactoe}

Um sich die Zeit während besonders langweiliger Vorlesungen bestmöglich
zu vertreiben, beschließen Sie eine einfache Tic-Tac-Toe-Implementierung
mit "`künstlicher Intelligenz"' zu schreiben. Die Regeln des Spiels bedürfen
wohl keiner weiteren Erklärung (diejenigen, die es nicht kennen, mögen einen
Blick auf \url{http://de.wikipedia.org/wiki/Tic_Tac_Toe} werfen). Das Augenmerk
sollte auf den Einsatz von Interprozesskommunikation, in dem Fall
mittels Shared Memory und Synchronisation mittels Semaphoren, gelegt werden.
Es sind also zwei Programme, \textit{Server} und \textit{Client}, zu
programmieren.

Die gesamte Spiellogik (und der einfache CPU-Gegner) sollen dabei auf Seite des
Serverprozesses implementiert werden -- der Client bietet dem Benutzer nur
ein praktisches Interface an und muss ledliglich das Setzen von Symbolen auf
bereits belegte Felder verbieten; um die Gewinnchancen etwas zu erhöhen, bietet
er auch die Möglichkeit, gleich zwei Symbole auf einmal zu platzieren (wovon
aber beide Plätze wieder frei sein müssen). Der Server sollte diesen Betrug
natürlich feststellen und dies unmittelbar als verloren für den Client werten!
Die Cheat-Erkennung kann auf Wunsch auch mit der Option \osuearg{-c} deaktiviert
werden.

\begin{verbatim}
    SYNOPSIS
        ttt-server [-c]
        ttt-client
\end{verbatim}

\subsection*{Details}
\subsubsection*{Server-Prozess}

Die Aufgaben des Servers nochmal zusammengefasst:

\begin{itemize}
\item Shared Memory und Semaphoren anlegen (und am Ende wieder freigeben!)
\item nach jedem Zug des Clients mittels einer lokalen Kopie des Spielfelds aufs
  Schummeln überprüfen (wenn dieser Modus nicht deaktiviert wurde!)
\item selbst ein Symbol an eine zufällige freie Position legen -- jedes der
  freien Felder soll dabei mit der selben Wahrscheinlichkeit gewählt werden!
\item auf Sieg/Verlust/Unentschieden prüfen und dies gegebenenfalls dem Client
  mitteilen
\end{itemize}

Der Server läuft in einer Endlosschleife und lässt sich lediglich per
\osuekeystroke{Ctrl+C} (\osueconst{SIGINT}) beenden. Achten Sie darauf, dass in
diesem Fall alle angelegten Ressourcen ordnungsgemäß wieder freigegeben werden!

\subsubsection*{Client-Prozess}
Der Client soll beim Starten zunächst versuchen, sich zum Server zu verbinden
(\emph{verbinden} heißt in dem Fall einfach zu überprüfen, ob die das Shared
Memory und die Semaphore, die der Server angelegen sollte, schon existieren --
falls sie nicht existieren, wird mit einer aussagekräftigen Fehlermeldung
abgebrochen). Bei erfolgreichem Verbindungsaufbau wird das aktuelle Spielfeld
angezeigt sowie eine Begrüßungsnachricht ausgegeben, mit dem Hinweis, dass mit
dem Befehl \osuecmd{h} weitere Hilfe zur Steuerung angefordert werden kann.

Dem Spieler stehen folgende Befehle zur Verfügung:

\begin{itemize}
	\item \osuecmd{s} -- Spielfeld anzeigen (dies wird außerdem nach jedem
	                     Zug automatisch gemacht)
	\item \osuecmd{p <xy>} -- Symbol platzieren auf Position \(xy\)
	                          (\(x, y \in \{1, 2, 3\}\) -- wobei $(1, 1)$
	                          die Ecke links oben bezeichnet)
	\item \osuecmd{c <xy> <xy>} -- zwei Symbole platzieren (Cheat-Modus)
	\item \osuecmd{n} -- Neues Spiel starten
	\item \osuecmd{q} -- Spiel/Client beenden (aktueller Spielstand bleibt
	                     weiterhin bestehen; der Client kann nach erneutem
	                     Aufruf das alte Spiel fortsetzen)
\end{itemize}

Falls ein nicht existierender Befehl eingegeben wurde, sollte eine entsprechende
Fehlermeldung ausgegeben und zusätzlich nochmal die Liste aller möglichen
Kommandos angezeigt werden. Falls der Befehl existiert, aber falsch verwendet
wurde (z.B.\ \osuecmd{p 45} oder \osuecmd{c 13}), sollte der Benutzer auch auf
den Fehler aufmerksam gemacht werden (z.B.\ \osueoutput{Error: field coordinates
out of bounds!}).

Beim Empfang eines besonderen Ereignisses vom Server (Gewinn, Verlust, Verlust
wegen Schummelns etc.) soll eine entsprechende Nachricht ausgegeben und nach
einem Tastendruck fortgesetzt werden. Der erste Zug pro Spiel soll immer vom
Client ausgeführt werden.

\subsection*{Hinweise}
Um eine Kommunikation zwischen Server und Client abseits des Spielfelds zu
ermöglichen (der Server muss den Client z.B.\ benachrichtigen, falls er verloren
oder gewonnen hat), ist es ratsam, außer dem Array noch eine Variable im Shared
Memory zu definieren, um Statusnachrichten auszutauschen. Dafür sind
\osuekeyword{struct}s ganz nützlich.

\osueguidelinesthree

\end{document}

