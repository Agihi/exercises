\input{../../template.ltx}

\begin{document}

\osuetitle{3}

\section*{Aufgabenstellung -- calendar}

Ganz knapp vor der Prüfung für \emph{Betriebssysteme}, auf die Sie sich seit
Monaten freuen, erfahren Sie, dass all Ihre Freunde dachten, die Prüfung fände
drei Tage später statt. Sofort evaluieren Sie deren Fehler und merken, das
Problem liegt am fehlenden Zeitmanagement Ihrer Kollegen. Hilfsbereit, wie Sie
sind, wollen Sie sie unterstützen und schreiben zu diesem Behufe eine simple
Kalenderapplikation.

\begin{verbatim}
    SYNOPSIS
        calserver [-v]
        calclient [-r idx] [-d offset] [entry]
\end{verbatim}

\subsection*{Details}

Server und Client sollen über Shared Memory miteinander kommunizieren. Der
Server verwaltet hierbei die Kalenderdaten; mit der Option \osuearg{-v} gibt er
Auskunft über die Aktionen, die er dabei durchführt.

Der Client fungiert gleichzeitig als Eingabe- und als Ausgabeapplikation. Soll
ein neuer Eintrag eingefügt werden, so wird \osueprog{calclient} mit
\osuearg{entry} als Text für diesen Eintrag aufgerufen. Ohne Angabe von
\osuearg{entry} sollen die Tageseinträge angezeigt werden. Mit dem Argument
\osuearg{offset} wird der zu bearbeitende Tag relativ zum heutigen angegeben;
beispielsweise würde \osuearg{-d 1} den morgigen Tag auswählen. Mit \osuearg{-r
idx} wird am ausgewählten Tag der Eintrag an Stelle \osuearg{idx} entfernt
oder, falls \osuearg{entry} mitangegeben wird, damit überschrieben.

\subsection*{Beispiele}

\begin{osuefmtcode}
$ \osueinput{./calclient}
Output:
1: Prf BSys
2: Abgabe Beispiel 1
\end{osuefmtcode}

Zeigt heutige Termine an.

\begin{osuefmtcode}
$ \osueinput{calclient -d 3 "VO BSys"}
\end{osuefmtcode}

Legt den Eintrag \osueinput{VO BSys} in drei Tagen an.

\begin{osuefmtcode}
$ \osueinput{./calclient -r 2 "Prf BSys"}
\end{osuefmtcode}

Ersetzt den zweiten Eintrag am heutigen Tag durch \emph{Prf BSys}.

\subsection*{Hinweise}
Überlegen Sie sich eine geeignete Datenstruktur zum Speichern der Einträge. Als
Schlüssel eignen sich u.a.\ UNIX-Timestamps (siehe dazu den Abschnitt \emph{The
Epoch} in der Manpage \osuefunc{time(7)}
bzw.\ \url{http://de.wikipedia.org/wiki/Unixzeit}), wobei sie dazu pro Tag einen
geeigneten Repräsentanten finden sollten. Sinnvolle Funktionen um damit
umzugehen sind \osuefunc{time(2)} und, falls notwendig, \osuefunc{mktime(3)}.

Vergessen Sie keinesfalls auf sprechende Fehlermeldungen.

\osueguidelinesthree

\end{document}

