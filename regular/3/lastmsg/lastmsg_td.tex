\input{../../template.ltx}

\begin{document}

\osuetitle{3}

\section*{Aufgabenstellung}


Man implentiere ein Programm dass den String eines Shared Memorys ausliest und
eventuell durch einen neuen ersetzt.
\begin{verbatim}
    SYNOPSIS
        lastmsg [-i id] [-c|-d] ["Message"]
\end{verbatim}

Die id (-i) steht für die ID des Shared Memory Segments (per default: die eigene Matrikelnummer).
Die ID des Semaphore ist die um eins erhöhte (default: Matrikelnummer+1).

Mit -c wird ein neues Shared Memory Segment und eine Semaphore angelegt.

Mit -d werden diese wieder gelöscht.

Das optionale Argument ''Message'' ist der in das Shared Memory zu schreibende String.

\section*{Anleitung}

Wird die Option -c übergeben, so soll ein neues Shared Memory Segment angelegt
werden - entweder mit der Default id (Matrikelnummer), oder mit einer über die Option -i
übergebenen id. Ausserdem soll eine Semaphore mit der um eins erhöhten id des
Shared Memory Segments angelegt werden (Matrikelnummer+1).

Die Option -d löscht Shared Memory und Semaphore wieder (auch hier ist die mit
-i übergebene id zu beachten).

Das Programm soll den im Shared Memory befindlichen String auf die Konsole
ausgeben (ausser -c wurde angegeben).

Wird eine ''Message'' übergeben, so soll das Programm diese in den Shared
Memory schreiben (ausser -d wurde angegeben).

\section*{Testen}

Mit folgendem Shell-Skript lässt sich testen ob sich das Programm korrekt
verhält:

\begin{verbatim}
	#!/bin/sh

	./lastmsg -c 1234
	 ( for i in `seq 1000`;do ./lastmsg foo & ./lastmsg bar & ./lastmsg blah & done; wait
	  ./lastmsg -d 1234 
	) > foo.tmp;
	grep -c foo foo.tmp
	grep -c bar foo.tmp
	grep -c blah foo.tmp
	rm -f foo.tmp
\end{verbatim}

Als Ergebnis soll drei mal ''1000'' ausgegeben werden. Weichen diese Werte
ab, so liegt vermutlich ein Synchronisierungsproblem vor.

\section*{Hinweise}

Die Optionen -c und -d schliessen sich wechselseitig aus.

\osueguidelinesthree

\end{document}

