\input{../../template.ltx}

\begin{document}

\osuetitle{3}

\section*{Aufgabenstellung -- lastmsg}

Man implementiere ein Programm, das den Inhalt eines Shared-Memory-Segments als
String ausliest und eventuell durch einen neuen String ersetzt.

\begin{verbatim}
    SYNOPSIS
        lastmsg [-i id] [-c|-d] ["Message"]
\end{verbatim}

Die \osuearg{id} (\osuearg{-i}) steht für die ID des Shared-Memory-Segments (per
default: die eigene Matrikelnummer). Die ID des Semaphores ist die um eins
erhöhte (default: Matrikelnummer + 1).

Mit \osuearg{-c} wird ein neues Shared-Memory-Segment und eine Semaphore
angelegt. Mit \osuearg{-d} werden diese wieder gelöscht.

Das optionale Argument \osuearg{\char`"Message\char`"} ist der in das Shared
Memory zu schreibende String.

\subsection*{Anleitung}

Wird die Option \osuearg{-c} übergeben, so soll ein neues Shared-Memory-Segment
angelegt werden -- entweder mit der voreingestellten \osuearg{id}
(Matrikelnummer), oder mit einer über die Option \osuearg{-i} übergebenen
\osuearg{id}. Außerdem soll eine Semaphore mit der um eins erhöhten \osuearg{id}
des Shared-Memory-Segments angelegt werden.

Die Option \osuearg{-d} löscht Shared Memory und Semaphore wieder (auch hier ist
die mit \osuearg{-i} übergebene \osuearg{id} zu beachten).

Das Programm soll den im Shared Memory befindlichen String auf die Konsole
ausgeben (außer \osuearg{-c} wurde angegeben).

Wird eine \osuearg{\char`"Message\char`"} übergeben, so soll das Programm diese
in den Shared Memory schreiben (außer \osuearg{-d} wurde angegeben).

\subsection*{Hinweise}

Die Optionen \osuearg{-c} und \osuearg{-d} schließen sich wechselseitig aus.
Außerdem darf keine \osuearg{\char`"Message\char`"} angegeben werden, wenn die
Option \osuearg{-d} gesetzt ist. Wird die Option \osuearg{-c} ohne
\osuearg{\char`"Message\char`"} angegeben, soll das neu angelegte Shared Memory
mit einem leeren String befüllt werden.

\subsection*{Testen}

Mit folgendem Shell-Skript lässt sich testen, ob sich das Programm korrekt
verhält:

\pagebreak[4]

\begin{verbatim}
#!/bin/sh
./lastmsg -c 1234

(
    for i in $(seq 1000)
    do
        ./lastmsg foo &
        ./lastmsg bar &
        ./lastmsg blah &
    done
    wait
    ./lastmsg -d
) > foo.tmp

grep -c foo foo.tmp
grep -c bar foo.tmp
grep -c blah foo.tmp
rm -f foo.tmp
\end{verbatim}

Als Ergebnis soll drei mal "`1000"' ausgegeben werden. Weichen diese Werte
ab, so liegt vermutlich ein Synchronisierungsproblem vor.

\osueguidelinesthree

\end{document}
