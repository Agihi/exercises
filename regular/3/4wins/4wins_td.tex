\input{../../template.ltx}

\begin{document}

\osuetitle{3}

\section*{Aufgabenstellung}

Wieder einmal wissen Sie nichts mit Ihrer Zeit anzufangen, wollen aber auch
nicht aus purer Langeweile eine Partie "`4 Gewinnt"' gegen sich selbst
verlieren. So implementieren Sie es, kurz entschlossen, einfach in C; und zwar
so, dass auch Sie eine Chance haben zu gewinnen.

\subsection*{Anleitung}

\begin{verbatim}
    SYNOPSIS:
        4wserver
        4wclient
\end{verbatim}

Schreiben Sie zwei Programme (\emph{Server} und \emph{Client}), die miteinander
über ein Shared Memory kommunizieren und sich mit Hilfe von Semaphoren
synchronisieren. Das Spiel besteht aus einem Server und nur einem Client, der
Server muss also nicht mehrere Clients gleichzeitig unterstützen. Das
(virtuelle) Spielfeld soll eine Größe von $7\times 6$ haben, und nur von oben
"`befüllbar"' sein (genau wie das Originalspiel). Der Server erzeugt das Shared
Memory und verwaltet eine interne Kopie vom enthaltenen Spielbrett, damit ein
Schummeln (mehrere Einwürfe oder unerlaubte Spielfeldveränderungen) erkennbar
ist. Außerdem soll der Server, als Computergegner, nach jedem gültigen Einwurf
des Clients selbst einwerfen; dabei sollen folgende Regeln berücksichtigt
werden:

\begin{itemize}
\item Mit einer Wahrscheinlichkeit von 1/3 soll dort eingeworfen werden, wo der
menschliche Gegner zuletzt eingeworfen hat.
\item Mit einer Wahrscheinlichkeit von 2/3 soll zufällig eingeworfen werden.
\item Ist die ausgewählte Spalte bereits voll, soll eine gefunden werden, die
noch Platz hat -- gibt es keine mehr, und keinen Gewinner, steht es
unentschieden.
\end{itemize}

Wird ein Sieg, Unentschieden oder Schummeln erkannt, gibt das der Server dem
Client bekannt; der Client zeigt bei einem Sieg oder Unentschieden nochmals das
Spielbrett an und startet das Spiel dann, nach Betätigung der Entertaste, neu.
Beim Schummeln soll das Spiel gleich neu gestartet werden, ohne dass das
Spielfeld nochmals angezeigt wird.

Der Client soll das Spielfeld anzeigen und über \osueglvar{stdin} in einer
Endlosschleife Kommandos entgegen nehmen. Nach jedem Kommando, das sich auf das
Spielfeld auswirkt, soll das Spielfeld erneut angezeigt werden. Arbeiten Sie mit
\osuefunc{system(3)} und dem Shell-Befehl \osueprog{clear}, um den Bildschirm zu
löschen.

\begin{verbatim}
Kommandos:

d<spalte nr>   wirf bei <spalte nr> ein, z.B. d2 oder d5
u<spalte nr>   nimm Zug bei <spalte nr> zurueck, z.B. u2 oder u5
c              beende Einwurf und gib das Spiel an den Computergegner zurueck
n              Neustart des Spiels
q              Beenden
\end{verbatim}

Beim Beenden des Clients soll das Spiel erhalten bleiben, sodass es beim
erneuten Start des Clients weitergespielt werden kann.

\osueguidelinesthree

\end{document}
