\input{../../template.ltx}
\begin{document}

\osuetitle{3}

\section*{Aufgabenstellung -- hangman}


In dieser Aufgabe implementieren Sie das Spiel "`Hangman"' in C. Die
Implementierung soll aus zwei Programme bestehen: ein Server, der die
Geheimwörter angibt sowie den Spielstand verwaltet, und ein Client, wo der
Benutzer versuchen kann Geheimwörter zu ermitteln.  Die
Interprozesskommunikation soll über ein Shared Memory Segment realisiert werden
und die Synchronisierung mittels Semaphoren erfolgen.


Der Client soll dem Spieler ein Interface bieten, das den aktuellen Spielstand
darstellt und dem Benutzer die Möglichkeit bietet Buchstaben des Geheimwortes
zu erraten.  Der Server soll angeben wie oft und an welcher Stelle des
Geheimwortes der geratene Buchstabe vorkommt. Falls der Buchstabe im Geheimwort
nicht enthalten ist, soll der Server einen weiteren Teil des Galgens mit einem
Gehängten zeichnen.

Es soll nicht zwischen Groß- und Kleinbuchstaben unterschieden werden. Wenn der
Spieler (Client) nach 9 Fehlern das Geheimwort noch nicht erraten hat, soll der
Galgen mit dem Gehängten komplett gezeichnet sein und der Spieler verliert.

\subsection*{Anleitung} Der Server muss eine beliebige Anzahl von Clients, die
gleichzeitig und unabhängig voneinander spielen können, unterstützen. Die
Kommunikation soll durch ein einziges Shared Memory Segment erfolgen
(\textbf{nicht} einem Shared Memory Segment pro Client).


\subsection*{Server}

\begin{verbatim}
USAGE: hangman-server [input_file]
\end{verbatim}

Der Server legt zu Beginn die benötigten Ressourcen an. Falls eine Eingabedatei
als Argument übergeben wird, liest der Server die Geheimwörter aus der Datei
\verb|input_file| ein. Beim Einlesen wird jede Zeile als ein Geheimwort
interpretiert, wobei alle Zeichen außer ASCII Buchstaben und Leerzeichen
ignoriert werden. Falls keine Eingabedatei angegeben wird, soll analog von der
Standardeingabe \verb|stdin| eingelesen werden, bis \verb|EOF| (\verb|Ctrl+D|)
auftritt.

Danach verwaltet der Server die Spiele der Clients und bedient auch deren
Anfragen.  Der Server muss dazu alle notwendige Informationen für jeden Client
verwalten. Insbesondere muss der Server alle Vorkommnisse der bisher korrekt
geratenen Buchstaben und die Anzahl an Fehlern den Clients in jeder Runde
bekannt geben.

Wenn das Spiel endet, soll der Server dem Client das Ergebnis übermitteln.
Wenn ein Client ein neues Spiel starten will, soll der Server das Spiel mit
einem bisher (für diesen Client) nicht verwendeten Geheimwort starten.  Falls
der Client ein Spiel mit allen Geheimwörtern gespielt hat, muss der Server den
entsprechenden Client darüber informieren.

Wenn ein registrierter Client aus irgendwelchen Gründen (inklusive Signale)
terminiert, soll der Server die entsprechenden Ressourcen aufräumen und
freigeben.  Zusatzlich muss der Server vor seiner eigenen Terminierung alle
verwendeten Ressourcen (z.B., dynamisch reservierter Speicher, Semaphoren,
Shared Memory, usw.) freigeben.  Der Server selbst soll durch die Signale
\verb|SIGINT| und \verb|SIGTERM| "`sauber"' terminierbar sein.



\subsection*{Client}

\begin{verbatim}
USAGE: hangman-client
\end{verbatim}

Der Client soll beim Starten zunächst versuchen sich zum Server zu verbinden
("`verbinden"' heißt hier zu überprüfen, ob die Semaphoren die der Server
anlegen sollte schon existieren).  Nach einem erfolgreichen Verbindungsaufbau
registriert sich der Client beim Server und kann ein neues Spiel beginnen.

Der Client erlaubt nun dem Spieler in jedem Spielzug einen bisher nicht
gewählten Buchstaben auswählen.  Falls das eingegebe Zeichen kein Buchstabe ist
oder schon einmal geraten wurde muss eine Fehlermeldung ausgegeben werden. Der
Client schickt den ausgewählten Buchstaben an den Server der den Spielzug
durchführt und den Spielstand als Antwort bereit stellt.

Der nach jedem Spielzug ausgegebene Spielstand muss mindestens die folgenden
Elemente enthalten:

\begin{itemize}
  \item Das Geheimwort, wobei erratene Buchstaben auf ihren entsprechenden
        Positionen im Wort großgeschrieben dargestellt werden, während die
        restlichen Buchstaben durch '\_' ersetzt werden.
        Leerzeichen werden von Anfang an dargestellt.
  \item Eine ASCII Zeichnung die ein anfaenglich unvollständiges Bild von einem
        Galgen mit Gehängten darstellt. Die Zeichnung muss nach jedem Fehler
        des Spielers erweitert und mit dem 9. Fehler vollständig werden.
  \item Eine Liste aller Buchstaben die der Spieler bisher in diesem Spiel
        geraten hat.
\end{itemize}

Nach Ende eines Spiels wird zuerst eine entsprechende Nachricht und der
aktuelle Win/Loss Stand (Anzahl der gewonnen und verlorerenen Spiele)
ausgegeben. Dann wird der Spieler gefragt, ob er mit dem nächsten Spiel
weitermachen will, worauf der Spieler mit ja/nein (y/n) antworten soll. Falls
er mit y antwortet, wird auf ein neues Spiel vom Server gewartet, sonst soll
der Client terminieren.

Falls der Server nach einem Spiel keine weiteren Spiele erstellen kann, also
der Client ein Spiel mit jedem Geheimwort des Servers gespielt hat, wird der
endgültige Win/Loss Stand ausgegeben und der Client beendet. Der Client kann
jederzeit mit den Signalen \verb|SIGINT| oder \verb|SIGTERM| beendet werden.
Außerdem soll der Client terminieren, wenn der Server beendet wurde.


Achten sie auf eine saubere Terminierung. Auch bei fehler- bzw. signalbezogener
Terminierung sollen die Clients und den Server sauber terminieren und alle
Ressourcen freigegeben werden.

\osueguidelinesthree

\end{document}
