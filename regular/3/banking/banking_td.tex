\input{../../template.ltx}

\begin{document}

\osuetitle{3}

\section*{Aufgabenstellung -- banking}

Es soll ein Programm geschrieben werden, dass die Geldtransfers der Kunden
verwaltet. Die Kunden können sich mit dem Programm verbinden und ihre
Geldgeschäfte erledigen. Das Verwaltungsprogramm der Bank ist der Server und
dieser kommuniziert mittels Shared Memory mit den Clients, die von den Kunden
benutzt werden.

Die Clients können Geld abheben und einzahlen, es auf anderes Konto überweisen
und den aktuellen Kontostand abfragen. Aus Gründen der Sicherheit und des
Datenschutzes müssen alle Informationen beim Server bleiben und die Clients
wissen nur ihre eigene Kontonummer. Außerdem darf pro Konto nur ein Client
angemeldet sein und jeder Client soll sich auch abmelden, wenn er sich beendet.



\subsection*{Anleitung}

Das Bank Account Management Tool soll als Client/Server Programm realisiert
werden, wobei ein Server beliebig viele Clients bedient. Die Kommunikation
zwischen Clients und Server soll per Shared Memory erfolgen.

\subsubsection*{Server}
\label{sec:server}

Der Server liest aus einer CSV-Datei die Liste der Konten ein. Jede Zeile ist
ein Konto und besteht aus der Kontonummer, dem Trennzeichen \texttt{';'} und
dem Kontostand. Wenn der Server beendet, gibt er den neuen Stand der Konten als
CSV-Datei aus.

\begin{verbatim}
USAGE: bankserver [ACCOUNT_LIST [OUT_ACCOUNT_LIST]]
\end{verbatim}

Zu Beginn liest der Server die Kontoliste ein und legt die benötigten
Ressourcen an. Falls eine Datei als Argument übergeben wurde, wird die Liste
aus dieser Datei gelesen, sonst wird die Liste von der Standardeingabe gelesen.
Ein EOF kennzeichnet das Ende der Eingabe.

Der Server soll beliebig viele Clients gleichzeitig bedienen können. Bei den
Anfragen, die von den Clients kommen, muss der Server zuvor prüfen, ob die
Operationen gültig sind, bevor Änderungen am Kontostand gemacht werden.
Ansonsten würde Geld verloren gehen und die Kunden wären unzufrieden.

Jeder Client muss sich vorher anmelden, bevor er Anfragen stellen kann.
Außerdem darf nicht mehr als ein Client pro Konto angemeldet sein. Der Server
muss dies überprüfen. Wenn sich ein Client vom Server trennt, muss sich dieser
davor abmelden. Ansonsten gilt der Client als angemeldet und es kann sich kein
weiterer Client für dieses Konto anmelden.

Der Server muss folgende Anfragen bearbeiten können: Abfragen des Kontostandes,
Abheben von Geld, Einzahlen von Geld, Überweisen auf ein anderes Konto. Beim
Abheben und Überweisen darf der Client danach keinen negativen Betrag am Konto
haben, d.h.\ es muss davor geprüft werden, ob genug Geld am Konto ist. Alle
Anfragen geben nur zurück, ob die Anfrage erfolgreich bearbeitet wurde oder
nicht. Bei der Abfrage des Kontostandes hingegen wird auch der Betrag an den
Client geschickt.

Beim Einlesen der Kontoliste darf der Kontostand negativ sein. Die Kontonummer
besteht aus sieben Dezimalstellen und die Kontonummer 0 ist reserviert und darf
von keinem Client benutzt werden. Für den Kontostand reicht eine
vorzeichenbehaftete 32 bit große Zahl.
Der Kontostand hat keine Nachkommastellen.

Falls der Server ein \verb|SIGTERM| oder \verb|SIGINT| Signal erhält, sollen
alle angelegten Ressourcen (lokal angelegter Speicher, Shared Memory,
Semaphore, etc) korrekt freigegeben und terminiert werden.


\subsubsection*{Client}

Der Client bekommt als Argument eine Kontonummer übergeben und versucht sich
mit dieser anzumelden.

\begin{verbatim}
USAGE: bankclient ACCOUNT_NR
\end{verbatim}

Der Client bietet dem Kunden ein einfaches Interface über die Standardeingabe
an, mit der der Kunde Anfragen an den Server schicken kann. Das Ergebnis jeder
Anfrage wird als Feedback auch auf die Standardausgabe ausgegeben.  Beim
Abfragen des Kontostandes wird auch der Betrag ausgegeben.

Beim Verbinden meldet sich der Client an und beim Trennen meldet er sich ab. Im
Falle eines Fehlers soll sich der Client so gut es geht, beim Server abmelden.
Der Client soll durch ein EOF an der Standardeingabe beendet werden.


\osueguidelinesthree

\end{document}
