% TeX source file
% Sysprog WS 2005
% Beispiel 2: chstat
% Dietmar Schabus (e0147322@student.tuwien.ac.at)
\input{../../template.ltx}

\begin{document}

\osuetitle{3}

\section*{Aufgabenstellung}

Implementieren Sie zwei Programme \osueprog{readin} und \osueprog{chstat}, die
zu einem Eingabetext eine Buchstabenstatistik erstellen.

\begin{verbatim}
SYNOPSIS:
     readin
     chstat [-v]
\end{verbatim} 




\subsection*{Anleitung}

Das Programm \osueprog{readin} liest von \osueglvar{stdin} und "ubergibt jede
gelesene Zeile "uber ein Shared Memory an das Programm \osueprog{chstat},
welches die Zeile liest und zu jedem Buchstaben speichert, wie oft er
vorgekommen ist. Gro"s-/Kleinschreibung soll dabei nicht beachtet
werden, es wird also zwischen A und a nicht unterschieden. Weiters
k"onnen Sie davon ausgehen, dass nur ASCII-Zeichen vorkommen, Umlaute
und dergleichen m"ussen Sie also nicht behandeln.

\osueprog{chstat} gibt dann eine Liste der Buchstaben A bis Z aus, jeweils
mit der Anzahl der Vorkommnisse sowie einer prozentuellen Angabe der
H"aufigkeit des Buchstabens (abgerundet auf ganze Prozent). Alle
anderen Zeichen werden in einer Kategorie ``andere''
zusammengefasst. Zus"atzlich wird die Gesamtanzahl der gelesenen
Zeichen ausgegeben.

Beispiel:
\begin{verbatim}
       A: 12    15%
       B: 6     7%
       C: 8     10%
       .
       .
       .
       Z: 0     0%
  andere: 24    30%
  gesamt: 80    100%
\end{verbatim}

Wird \osueprog{chstat} mit der Option \osuearg{-v} aufgerufen, gibt es nach
jeder gelesenen Zeile die aktualisierte Statistik aus, ohne \osuearg{-v}
wird nur die endg"ultige Statistik "uber alle Zeilen ausgegeben.

\osueprog{chstat} soll die n"achste Zeile erst dann lesen, wenn \osueprog{readin}
sie vollst"andig ins Shared Memory geschrieben hat und \osueprog{readin} soll
erst dann die n"achste Zeile ins Shared Memory schreiben,
wenn die vorherige schon von \osueprog{chstat} gelesen wurde.

Durch die Eingabe von \osueconst{EOF} (also durch Dr"ucken von \osuekeystroke{Ctrl-D})
wird das Ende des Textes angezeigt und beide Programme terminieren.



\subsection*{Testen}

Die Option \osuearg{-v} von \osueprog{chstat} eignet sich gut zum interaktiven
Testen, da die Auswirkungen einer neuen Zeile auf die Statistik sofort
sichtbar sind, das Testen ohne \osuearg{-v} ist f"ur die Umleitung von
\osueglvar{stdin} auf ein Testfile vielleicht sinnvoller:
\begin{verbatim}
  readin < mytestfile
\end{verbatim}

\osueguidelinesthree

\end{document}
