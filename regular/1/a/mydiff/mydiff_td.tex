\input{../../../template.ltx}

\begin{document}

\osuetitle{1}

\section*{Aufgabenstellung A}
Implementieren Sie eine Abwandlung des Unix-Kommandos \osueprog{diff}.
Schreiben Sie zu diesem Zweck ein C-Programm
\osueprog{mydiff}, dass 2 übergebene Dateien zeilenweise
miteinander vergleicht und, falls sich 2 Zeilen voneinander
unterscheiden, die Zeilennummer und die Anzahl der
verschiedenen Zeichen ausgibt.

\begin{verbatim}
	SYNOPSIS
      		mydiff file1 file2
\end{verbatim}


\subsection*{Anleitung}
      Das Programm soll jede Datei zeilenweise einlesen und die
      gelesenen Zeichen miteinander vergleichen. Sind die Zeilen
      ungleich lang, so soll nur bis zur Länge der kürzeren Zeile
      verglichen werden - z.B.: \osueinput{,,Haus$\backslash$n''} und \osueinput{,,Haustor$\backslash$n''}
      sind als gleich zu behandeln. Falls eine Datei mehr Zeilen enthält
      als die andere, sollen die restlichen Zeilen ebenfalls ignoriert
      werden. Pro Zeile soll gezählt werden, wieviele Zeichen an der
      gleichen Position nicht übereinstimmen.


      Definieren Sie für die maximale Anzahl an Zeichen in einer
      Zeile eine Konstante, wobei Sie annehmen dürfen, dass keine
      Zeile mehr Zeichen enthält.

\subsection*{Testen}
Erstellen Sie eine Testdatei \osuefilename{difftest1.txt} mit folgendem Inhalt:
\subsubsection*{File: difftest1.txt}
\begin{verbatim}
      abcdefg
      Garten
      abcdefg
\end{verbatim}
      und eine zweite Testdatei \osuefilename{difftest2.txt} mit
      folgendem Inhalt:
\subsubsection*{File: difftest2.txt}
\begin{verbatim}
      abcdefg
      Gartenzaun
      ahciejg
      abcdefg
\end{verbatim}
    Rufen Sie Ihr Programm mit folgenden Argumenten auf:
\begin{verbatim}
    $ mydiff difftest1.txt difftest2.txt
    Zeile: 3 Zeichen: 3
\end{verbatim}
\begin{verbatim}
    $ mydiff difftest1.txt bloedsinn.txt
    mydiff: File bloedsinn.txt existiert nicht!
\end{verbatim}

\osueguidelinesone

\end{document}
 
