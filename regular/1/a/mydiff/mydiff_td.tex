\input{../../../template.ltx}

\begin{document}

\osuetitle{1}

\section*{Aufgabenstellung A -- mydiff}
Implementieren Sie eine Abwandlung des Unix-Kommandos \osueprog{diff}.
Schreiben Sie zu diesem Zweck ein C-Programm
\osueprog{mydiff}, das zwei übergebene Dateien zeilenweise
miteinander vergleicht und, falls sich zwei Zeilen voneinander
unterscheiden, die Zeilennummer und die Anzahl der
verschiedenen Zeichen ausgibt.

\begin{verbatim}
    SYNOPSIS:
        mydiff [-i] file1 file2
\end{verbatim}

\subsection*{Anleitung}
      Das Programm soll jede Datei zeilenweise einlesen und die
      gelesenen Zeichen miteinander vergleichen. Sind die Zeilen
      ungleich lang, so soll nur bis zur Länge der kürzeren Zeile
      verglichen werden -- z.B.: \verb+Haus\n+ und \verb+Haustor\n+
      sind als gleich zu behandeln. Falls eine Datei mehr Zeilen enthält
      als die andere, sollen die restlichen Zeilen ebenfalls ignoriert
      werden. Pro Zeile soll gezählt werden, wieviele Zeichen an der
      gleichen Position nicht übereinstimmen. Falls die Option \verb|-i|
      angegeben ist, soll Groß-/Kleinschreibung nicht als Unterschied
      gezählt werden.

      Definieren Sie für die maximale Anzahl an Zeichen in einer
      Zeile eine Konstante, wobei Sie annehmen dürfen, dass keine
      Zeile mehr Zeichen enthält.

\subsection*{Testen}

Erstellen Sie eine Testdatei \osuefilename{difftest1.txt} mit folgendem Inhalt:

\begin{osuefmtcode}
      abcdefg
      Garten
      abcdefg
\end{osuefmtcode}

und eine zweite Testdatei \osuefilename{difftest2.txt} mit folgendem Inhalt:

\begin{osuefmtcode}
      Abcdefg
      Gartenzaun
      ahciejg
      abcdefg
\end{osuefmtcode}

Rufen Sie Ihr Programm dann mit folgenden Argumenten auf:

\begin{osuefmtcode}
    $ \osueinput{./mydiff -i difftest1.txt difftest2.txt}
    Zeile: 3 Zeichen: 3
    $ \osueinput{./mydiff difftest1.txt bloedsinn.txt}
    mydiff: Datei bloedsinn.txt existiert nicht!
\end{osuefmtcode}

\osueguidelinesone

\end{document}
