\input{../../../template.ltx}

\begin{document}

\osuetitle{1}

\section*{Aufgabenstellung A -- mycompress}
      Schreiben Sie ein C-Programm \osueprog{mycompress},
      das die als Argumente übergebenen Eingabedateien mittels
      eines einfachen Algorithmus komprimiert.
      Wird keine Eingabedatei angegeben, so ist von
      \osueglvar{stdin} zu lesen.
\begin{verbatim}
    SYNOPSIS:
        mycompress [-o outfile] [infile1] [infile2] ...
\end{verbatim}

\subsection*{Anleitung}
      Das Programm soll den Inhalt der übergebenenen Eingabedateien
      nacheinander auslesen, komprimieren, und die komprimierte Form in ein
      File mit dem durch die Option \verb|-o| spezifizierten Namen
      \osuefilename{outfile} ausgeben.
      Werden keine Eingabedateien angegeben, so ist von \osueglvar{stdin} zu
      lesen.
      Fehlt die Option \verb|-o|, wird auf \osueglvar{stdout} ausgegeben.

      Die Komprimierung soll so erfolgen, dass die Zeichen durch
      \verb|Zeichen + Anzahl| ersetzt werden -- z.B.: \verb|aaa| durch
      \verb|a3| und \verb|b| durch \verb|b1|.
      Das Newline ist auch ein Zeichen und soll ebenfalls komprimiert werden.

      Geben Sie die Anzahl der gelesenen Zeichen, die Anzahl der geschriebenen
      Zeichen und die daraus resultierende Komprimierungsrate
      auf die Standardfehlerausgabe \osueglvar{stderr} aus.
      (Sie werden feststellen, dass die Komprimierung nur bei vielen gleichen
      Zeichen effizient ist.)

      Definieren Sie für die maximale Anzahl an Zeichen in einer
      Zeile eine Konstante, wobei Sie annehmen dürfen, dass keine
      Zeile mehr Zeichen enthält.

\subsection*{Testen}
      Testen Sie Ihr Programm mit verschiedenen Eingaben; z.B.\ soll
      die Eingabe

\begin{osuefmtcode}
      aaabbbbbc
      dddddddde
      ggghhhhha
\end{osuefmtcode}

die Ausgabe

\begin{osuefmtcode}
      a3b5c1
      1d8e1
      1g3h5a1
      1
\end{osuefmtcode}

auf \osueglvar{stdout} (bzw.\ der angegebenen Ausgabedatei) und die Ausgabe

\begin{osuefmtcode}
      Gelesen:          30 Zeichen
      Geschrieben:      22 Zeichen
      Komprimierungsrate:   73.3 \%
\end{osuefmtcode}

auf \osueglvar{stderr} liefern.


\osueguidelinesone

\end{document}

