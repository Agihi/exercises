\input{../../../template.ltx}

\begin{document}

\osuetitle{1}

\section*{Aufgabenstellung A}
       Schreiben Sie ein C-Programm \emph{mycompress},
      das die als Argumente übergebenen Dateien mittels
      eines einfachen Algorithmus komprimiert.
      Wird kein Argument angegeben, so ist von
      \emph{stdin} zu lesen.
\begin{verbatim}
	 SYNOPSIS
        mycompress [file1] [file2] ...
\end{verbatim}

\subsection*{Anleitung}
      Das Programm soll den Inhalt des übergebenen Files
      auslesen, komprimieren und nach der Komprimierung in ein
      File mit dem Namen \emph{[ursprüngliches File].comp}
      ausgeben. Wird kein File übergeben, so ist von stdin zu
      lesen und in das File \emph{Stdin.comp} auszugeben. Die
      Komprimierung soll so erfolgen, dass die Zeichen durch
      \verb|Zeichen + Anzahl| ersetzt werden - z.B.: \verb|aaa| durch \verb|a3|
      und \verb|b| durch \verb|b1|. 
	Newline ist auch ein Zeichen und soll ebenfalls mit [zeichen][anzahl] komprimiert werden.
      
      Geben Sie die Anzahl der Zeichen - auch \emph{newline}
      sowohl des unkomprimierten, als auch des komprimierten Files
      auf dem Bildschirm aus (Sie werden feststellen, dass die
      Komprimierung nur bei vielen gleichen Zeichen effizient ist).

 	
      
      Definieren Sie für die maximale Anzahl an Zeichen in einer
      Zeile eine Konstante, wobei Sie annehmen dürfen, dass keine
      Zeile mehr Zeichen enthält.

      
\subsection*{Testen}
      Testen Sie Ihr Programm mit verschiedenen Textfiles, z.B
      soll eine Testdatei \emph{Test.txt} mit folgenden Inhalt:
\subsubsection*{File: Test.txt}
\begin{verbatim}
      aaabbbbbc
      dddddddde
      ggghhhhha
\end{verbatim}
die Bildschirmausgabe:
\begin{verbatim}
      Test.txt:        30 Zeichen
      Test.txt.comp:   22 Zeichen
\end{verbatim}

und die Outputdatei:

\subsubsection*{File: Test.txt.comp}
\begin{verbatim}
      a3b5c1
      1d8e1
      1g3h5a1
      1
\end{verbatim}

liefern.

\osueguidelinesone

\end{document}

