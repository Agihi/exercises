\input{../../../template.ltx}

\begin{document}

\osuetitle{1}

\section*{Aufgabenstellung A}
      Schreiben Sie ein C-Programm \osueprog{mycompress},
      das die als Argumente übergebenen Dateien mittels
      eines einfachen Algorithmus komprimiert.
      Wird kein Argument angegeben, so ist von
      \osueglvar{stdin} zu lesen.
\begin{verbatim}
    SYNOPSIS:
        mycompress [file1] [file2] ...
\end{verbatim}

\subsection*{Anleitung}
      Das Programm soll den Inhalt der übergebenen Datei
      auslesen, komprimieren, und nach der Komprimierung in ein
      File mit dem Namen \osuefilename{[ursprünglicher Name].comp}
      ausgeben. Wird keine Datei übergeben, so ist von \osueglvar{stdin} zu
      lesen und in die Datei \osuefilename{Stdin.comp} auszugeben. Die
      Komprimierung soll so erfolgen, dass die Zeichen durch
      \verb|Zeichen + Anzahl| ersetzt werden -- z.B.: \verb|aaa| durch \verb|a3|
      und \verb|b| durch \verb|b1|.

      Das Newline ist auch ein Zeichen und soll ebenfalls komprimiert werden.

      Geben Sie die Anzahl der Zeichen -- auch Newlines --
      sowohl der unkomprimierten, als auch der komprimierten Datei
      auf dem Bildschirm aus. (Sie werden feststellen, dass die
      Komprimierung nur bei vielen gleichen Zeichen effizient ist.)

      Definieren Sie für die maximale Anzahl an Zeichen in einer
      Zeile eine Konstante, wobei Sie annehmen dürfen, dass keine
      Zeile mehr Zeichen enthält.

\subsection*{Testen}
      Testen Sie Ihr Programm mit verschiedenen Textdateien; z.B.\ soll
      eine Testdatei \osuefilename{Test.txt} mit folgenden Inhalt:

\begin{osuefmtcode}
      aaabbbbbc
      dddddddde
      ggghhhhha
\end{osuefmtcode}

die Bildschirmausgabe:

\begin{osuefmtcode}
      Test.txt:        30 Zeichen
      Test.txt.comp:   22 Zeichen
\end{osuefmtcode}

und die Outputdatei \osuefilename{Test.txt.comp} mit dem Inhalt:

\begin{osuefmtcode}
      a3b5c1
      1d8e1
      1g3h5a1
      1
\end{osuefmtcode}

liefern.

\osueguidelinesone

\end{document}

