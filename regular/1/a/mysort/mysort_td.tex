\input{../../../template.ltx}

\begin{document}

\osuetitle{1}

\section*{Aufgabenstellung A}
Schreiben Sie eine abgewandelte Version des UNIX-Kommandos {\bf sort} als C-Programm. Es soll nur die Option \verb|-r| (absteigend sortieren) implementiert werden. Alle anderen Optionen können ignoriert werden. Wird keine Datei angegeben so sollen die Daten \"uber \emph{stdin} eingelesen werden.

\begin{verbatim}
	SYNOPSIS
	  mysort [-r] [file1] ...
\end{verbatim}

\subsection*{Anleitung}
Lesen Sie alle Files zeilenweise in einen dafür geeigneten Puffer ein.
Danach sortieren Sie die Daten mit Hilfe von \emph{qsort(3)}. Im Anschluss geben Sie die nun sortieren Daten auf \emph{stdout} aus. Es kann davon ausgegangen werden, daß keine Zeile länger als 1022 Zeichen ist (ohne Newline).

\subsection*{Testen}
Testen Sie Ihr Programm mit verschieden Eingaben. Erstellen Sie zum Beispiel ein Testfile \emph{t1} mit folgenden Zeilen.

\subsubsection*{File: t1}
\begin{verbatim}
  Priority9 cat
  Priority2 ls
  Priority7 cat mysort.h
\end{verbatim}
Aufruf: {\bf cat t1 $|$ mysort} oder {\bf mysort $<$ t1} \\
Ausgabe: 
\begin{verbatim}
  Priority2 ls
  Priority7 cat mysort.h
  Priority9 cat
\end{verbatim}


Aufruf: {\bf mysort -r t1 t1} \\
Ausgabe: 
\begin{verbatim}
  Priority9 cat
  Priority9 cat
  Priority7 cat mysort.h
  Priority7 cat mysort.h
  Priority2 ls
  Priority2 ls
\end{verbatim}

\subsection*{Hinweis}
{\bf { ( echo a ; echo B )} $|$ sort} sortiert B vor a, weil die Großbuchstaben im Zeichensatz vor den Kleinbuchstaben kommen. {\bf { ( echo a ; echo B )} $|$ sort -f} würde diese Eigenschaft ignorieren sein. Die Option \verb|-f| braucht aber {\bf nicht} implementiert werden.

\osueguidelinesone

\end{document}
