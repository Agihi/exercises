\input{../../../template.ltx}

\begin{document}

\osuetitle{1}

\section*{Aufgabenstellung A}

Eines kalten Herbstabends finden Sie am Bildschirm eine seltsame Botschaft, die
Ihnen erklärt, dass Sie auserwählt wurden.

Beeindruckt beschließen Sie den Anweisungen darin sogleich Folge zu leisten --
bevor Sie überhaupt noch wissen, worum es geht. Der Absender behauptet wichtige
Informationen für Sie zu haben, benötigt aber schnellstens ein Tool, um Ihnen
die geheime, möglicherweise existenzverändernde Botschaft zuschicken zu können.
Sie erkennen sofort die Wichtigkeit der Situation und machen sich aufgeregt an
die Implementierung des Programms \emph{stegit}, um raschest die geheime
Botschaft lesen zu können.

\begin{verbatim}
          SYNOPSIS: stegit -f|-h [-o <filename>]

           -f                 find mode
           -h                 hide mode
          [-o <filename>]     output filename
\end{verbatim}

Dieses Programm soll in zwei Modi ausgeführt werden können: \emph{Verstecken}
und \emph{Finden}.

Im Modus \emph{Verstecken} wird solange von \emph{stdin} gelesen, bis \emph{EOF}
auftritt oder die Enter-Taste betätigt wird. Dann wird die eingelesene Botschaft
in einem Text versteckt und auf \emph{stdout} oder in die Ausgabedatei
ausgegeben. Im Modus \emph{Finden} soll wiederum von \emph{stdin} so lange
gelesen werden, bis \emph{EOF} auftritt. Aus dem eingelesenen Text soll die
geheime Botschaft ermittelt werden und entweder auf \emph{stdout} angezeigt oder
in die Ausgabedatei geschrieben werden. Falls eine Ausgabedatei angegeben wurde,
muss dafür gesorgt werden, dass sie überschrieben wird, wenn sie schon vorhanden
war; ansonsten soll sie neu angelegt werden.

\subsection*{Anleitung}
Suchen Sie sich 28 beliebige Wörter und initialisieren Sie damit ein
Zeichenketten-Array. Dieses soll sich fix im Programm befinden und nicht
veränderbar sein. Mit Hilfe der Array-Indizes weisen Sie jedem Buchstaben im
englischen Alphabet, und den Sonderzeichen \emph{Leerzeichen} und \emph{Punkt},
ein Wort zu. Damit sollte es kein Problem mehr sein, die eingelesene Botschaft
in (unter Umständen) unsinnigen Text umzuwandeln. Sonderzeichen und nicht
zugeordnete Zeichen werden ignoriert (geschluckt). Damit der resultierende
End-Text "`realistischer"' aussieht, bauen Sie zufällig (nach je 5-15 Wörtern)
Punkte ein -- die Sie beim Einlesen im Find-Modus natürlich wiederum ignorieren.
Die maximale Länge der geheimen Botschaft (Cleartext) können Sie auf 300 Zeichen
begrenzen.

Beispieltabelle:
\begin{verbatim}
p -> "der"
s -> "Himmel"
c -> "ist"
h -> "heute"
t -> "klar"
. -> "la"
\end{verbatim}
Die Geheimbotschaft:
\begin{verbatim}
pscht...
\end{verbatim}
würde nach der Hide-Operation in folgenden Text resultieren (der Punkt ist
Zufall):
\begin{verbatim}
der himmel ist heute klar. la la la
\end{verbatim}

\subsection*{Testen}
Starten Sie das Programm:
\begin{verbatim}
stegit -h -o secret_message_within
\end{verbatim}
und geben Sie folgenden Text ein (möglichst ohne dass Sie mitlesen!):
\begin{verbatim}
es muessen beide aufgabenstellungen a und b geloest werden
\end{verbatim}
Drücken Sie die Enter-Taste und betrachten Sie die nun versteckte Botschaft mit:
\begin{verbatim}
cat secret_message_within
\end{verbatim}
Versuchen Sie aus dem entstandenen Text mit:
\begin{verbatim}
stegit -f < secret_message_within
\end{verbatim}
die geheime Botschaft wiederherzustellen. Sie könnte Sie vor bösen
Überraschungen bewahren.

\osueguidelinesone

\end{document}
