\input{../../../template.ltx}

\begin{document}

\osuetitle{1}

\section*{Aufgabenstellung A -- myexpand}
Implementieren Sie eine vereinfachte Variante des Unix-Kommandos
\osueprog{expand}.
\begin{verbatim}
    SYNOPSIS:
        myexpand [-t tabstop] [file...]
\end{verbatim}

Das Programm \osueprog{myexpand} soll die als Argumente angegebenen Dateien
lesen. Ist keine Datei angegeben, soll von \osueglvar{stdin} gelesen werden.
Dabei werden auftretende Tabs durch Leerzeichen ersetzt. Die Ausgabe
soll auf \osueglvar{stdout} erfolgen. Der optionale Parameter \osuearg{tabstop}
gibt an, an welchen Positionen die Tabs enden sollen (fehlt dieser,
ist der Wert 8 anzunehmen).

\subsection*{Anleitung}
Lesen Sie die Dateien zeichenweise ein und überprüfen Sie den
ASCII-Code des eingelesenen Zeichens. Handelt es sich um ein Tab
(\verb|\t|), berechnen Sie die Position \verb|p| des folgenden
Zeichens als nächstes Vielfaches von \verb|tabstop| größer der
aktuellen Position plus 1:

\verb|p = tabstop * ((x / tabstop) + 1)|

wobei \verb|x| die Position des Tabs in der aktuellen Zeile und
\verb|/| eine ganzzahlige Division (mit Abschneiden der
Nachkommastellen) beschreibt.

\subsection*{Testen}
Testen Sie Ihr Programm mit mehreren Eingabedateien. Erstellen Sie
z.B. eine Testdatei \osuefilename{t1} mit folgendem Inhalt (\verb|\t| steht für
ein Tab-Zeichen):

\begin{osuefmtcode}
	1234567890
	123\bslash{}t90
\end{osuefmtcode}

Rufen Sie Ihr Programm dann wie folgt auf:

\begin{osuefmtcode}
$ \osueinput{./myexpand t1}
	1234567890
	123     90

$ \osueinput{cat t1 \textup{|} ./myexpand}
	1234567890
	123     90

$ \osueinput{./myexpand -t 6 t1}
	1234567890
	123   90
\end{osuefmtcode}

\osueguidelinesone

\end{document}
