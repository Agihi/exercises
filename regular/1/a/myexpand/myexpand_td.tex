\documentclass{article}
\usepackage[german]{babel}
\usepackage[utf8]{inputenc}
\usepackage{a4wide}
\pagestyle{empty}
\parindent0pt

\begin{document}

\begin{center}
\begin{Large}
OPERATING SYSTEMS BEISPIEL 1
\end{Large}
\end{center}

\section*{Aufgabenstellung A}
Implementieren Sie eine vereinfachte Variante des Unix-Kommandos {\tt
expand}.
\begin{verbatim}
    SYNOPSIS:
        myexpand [-t tabstop] [file...]
\end{verbatim}

Das Programm {\tt myexpand} soll die als Argumente angegebenen Dateien
lesen. Ist keine Datei angegeben, soll von {\it stdin} gelesen werden.
Dabei werden auftretende Tabs durch Leerzeichen ersetzt. Die Ausgabe
soll auf {\it stdout} erfolgen. Der optionale Parameter \verb|tabstop|
gibt an, an welchen Positionen die Tabs enden sollen (fehlt dieser,
ist der Wert 8 anzunehmen).

\subsection*{Anleitung}
Lesen Sie die Dateien zeichenweise ein und überprüfen Sie den
ASCII-Code des eingelesenen Zeichens. Handelt es sich um ein Tab
(\verb|\t|), berechnen Sie die Position \verb|p| des folgenden
Zeichens als nächstes Vielfaches von \verb|tabstop| größer der
aktuellen Position plus 1:

\verb|p = tabstop * ((x / tabstop) + 1)|

wobei \verb|x| die Position des Tabs in der aktuellen Zeile und
\verb|/| eine ganzzahlige Division (mit Abschneiden der
Nachkommastellen) beschreibt.

\subsection*{Testen}
Testen Sie Ihr Programm mit mehreren Eingabedateien. Erstellen Sie
z.B. eine Testdatei t1 mit folgendem Inhalt (\verb|\t| steht für einen
Tab-Character):
\begin{verbatim}
	1234567890
	123\t90
\end{verbatim}

Befehl: \verb!myexpand t1! oder \verb!cat t1 | myexpand!

Ausgabe:
\begin{verbatim}
	1234567890
	123     90
\end{verbatim}

Befehl: \verb!myexpand -t 6 t1!

Ausgabe:
\begin{verbatim}
	1234567890
	123   90
\end{verbatim}

\end{document}
