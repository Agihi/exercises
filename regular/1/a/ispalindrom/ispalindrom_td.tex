\input{../../../template.ltx}

\begin{document}

\osuetitle{1}

\section*{Aufgabenstellung A -- ispalindrom}

Implementieren Sie ein Programm \osueprog{ispalindrom}, welches
eingegebene Strings auf Palindromeigenschaften überprüft.

\begin{verbatim}
    SYNOPSIS:
        ispalindrom [-s] [-i]
\end{verbatim}

Das Programm \osueprog{ispalindrom} soll zeilenweise Strings mit einer
Länge von bis zu 40 echten Zeichen von der Standardeingabe lesen;
prüfen, ob ein Palindrom vorliegt, d.h.\ ob der Text rückwärts gelesen
mit sich selbst ident ist; und den Text gefolgt von \osueoutput{ist ein
Palindrom} bzw. \osueoutput{ist kein Palindrom} auf die Standardausgabe
ausgeben. Die Option \osuearg{-s} soll bewirken, dass Leerzeichen
ignoriert werden; die Option \osuearg{-i} soll bewirken, dass nicht
zwischen Groß- und Kleinschreibung unterschieden wird.

\subsection*{Testen}

Testen Sie Ihr Programm mit verschiedenen Eingaben, wie z.B.:

\begin{osuefmtcode}
      $ \osueinput{./ispalindrom}
      \osueinput{Reliefpfeiler}
      Reliefpfeiler ist kein Palindrom
      \osueinput{reliefpfeiler}
      reliefpfeiler ist ein Palindrom

      $ \osueinput{./ispalindrom -i -s}
      \osueinput{Reliefpfeiler}
      Reliefpfeiler ist ein Palindrom
      \osueinput{O Genie der Herr ehre Dein Ego}
      O Genie der Herr ehre Dein Ego ist ein Palindrom
      \osueinput{aaaaaaaaaaaaaaaaaaaaaaaaaaaaaaaaaaaaaaaaaaaaaaaaaaaa}
      ispalindrom: Eingabe zu lang, max 40 Zeichen!
\end{osuefmtcode}

\osueguidelinesone

\end{document}
