\documentclass{article}
\usepackage[utf8]{inputenc}
\usepackage[german]{babel}
\usepackage{bytefield}
\usepackage{a4wide}
\usepackage{nameref}
\parindent0pt
\pagestyle{empty}


\begin{document}
\begin{center}
\begin{Large}
OPERATING SYSTEMS BEISPIEL 1
\end{Large}
\end{center}


\section*{Aufgabenstellung B -- Mastermind}\label{sec:aufgabenstellung}
Implementieren Sie einen Client und einen Server, die mittels TCP/IP
miteinander kommunizieren. Dabei soll der Client den Spieler und der Server den
Spielleiter des Spiels ``Mastermind'' implementieren. In diesem Spiel versucht
der Spieler eine vom Spielleiter geheimgehaltene, geordnete Folge von Farben zu
erraten. \emph{Beachten Sie, dass der Spieler vollautomatisch agieren soll}. Daher
ist die Implementierung einer Spielstrategie ebenfalls Teil der Aufgabe
(siehe Abschnitt~\nameref{sec:grading}).

In der hier verwendeten Variante des Spiels besteht eine Folge aus 5 Farben,
wobei folgende 8 Farben zur Verfügung stehen:
\textbf{b}eige,
\textbf{d}unkelblau,
\textbf{g}r\"un,
\textbf{o}range,
\textbf{r}ot,
\textbf{s}chwarz,
\textbf{v}iolett,
\textbf{w}eiß.

Nachdem eine Verbindung zwischen dem Client und dem Server hergestellt wurde,
wird sofort mit dem Spiel begonnen. Der Spieler sendet die von ihm vermutete
korrekte Folge von Farben an den Spielleiter. Dieser antwortet mit der Anzahl
korrekt positionierter Farben (die Anzahl roter Stifte im Brettspiel), und der
Anzahl an Farben, die zusätzlich ebenfalls in der korrekten Lösung enthalten
sind, jedoch an der falschen Position vermutet wurden (Anzahl \emph{weißer}
Stifte). Beachten Sie dabei, dass die Summe roter und weißer Stifte für eine
Farbe die Kardinalität dieser Farbe in der korrekten Lösung nicht übersteigen
kann.

Hierzu ein Beispiel: Angenommen die korrekte Lösung ist die Folge 
\begin{verbatim}
  rot, rot, grün, grün, grün
\end{verbatim}
und der Spieler vermutet bei der korrekten Folge handle es sich um
\begin{verbatim}
  rot, grün, rot, beige, rot
\end{verbatim}
Dann ist lediglich die erste Farbe der vermuteten Lösung an der korrekten
Position (ein roter Stift), und \emph{zwei} weitere Elemente der Folge (einmal
Farbe grün und \emph{einmal} Farbe rot) sind zusätzlich an einer anderen
Position in der korrekten Lösung enthalten.

Das Spiel endet, wenn der Spieler die korrekte Folge erraten hat, der Server
einen Protokollfehler meldet, oder die maximale Anzahl an Runden (\textbf{35})
erreicht wurde. Am Ende des Spiels sollen sowohl Server als auch Client
entweder die Anzahl gespielter Runden, oder den vom Server übermittelten
Fehler ausgeben.

\subsection*{Implementierungshinweise}
\label{sec:implhints}
\paragraph{Server:}
Teile des Servers sind bereits vorgegeben. Bitte verwenden Sie dieses Template und erweitern Sie es entsprechend.
Dem Server wird als erstes Argument der Port übergeben, auf dem er für die Clients
erreichbar sein soll. Das zweite Argument ist ein String der Länge \textbf{5},
bestehend aus den Anfangsbuchstaben der 5 Farben der geheimen Zahlenfolge.

%  Erstellen Sie mittels \texttt{socket(AF\_INET,SOCK\_STREAM,0)} einen TCP/IP
% Socket und binden sie diesen Socket and die übergeben Portnummer mittels
% \emph{bind}. Bereiten sie den Socket mit dem Kommando \emph{listen} auf das
% Akzeptieren von Verbindungen vor, und rufen Sie schließlich in einer
% Endlosschleife \emph{accept} auf, um Verbindungsanfragen der Clients zu
% akzeptieren.

Der Server soll auf eine eingehende Verbindungen warten. Sobald eine
Verbindung akzeptiert wurde, beginnt ein neues Spiel, und der Server
beantwortet bis zum Ende des Spiels die Anfragen des Clients. Am Ende des
Spiels werden entweder die Anzahl der gespielten Runden oder der zuletzt
übermittelte Fehler ausgegeben, und das Serverprogramm wird mit einem
dementsprechenden Rückgabewert (siehe weiter unten) beendet. Sobald der Server
\texttt{SIGINT}, \texttt{SIGQUIT} oder \texttt{SIGTERM} empfängt, soll der
Serversocket geschlossen, und das Programm verlassen werden (Rückgabewert 0).

\begin{verbatim}
Server: SYNOPSIS
   server <server-port>  <secret-sequence>
        EXAMPLE
   server 1280 wwrgb
\end{verbatim}

\paragraph{Client:}

Dem Client wird beim Aufruf der Hostname und die Portnummer des Servers
übergeben (wenn der Client auf der selben Maschiene wie der Host ausgeführt
wird, dann geben sie \emph{localhost} für den Hostnamen des Servers an). Legen
sie zuerst einen TCP/IP Socket an. Stellen Sie dann die zum Hostnamen des
Servers zugehörige IP-Adresse fest, und verbinden Sie sich mit dem Server.
Danach wird sofort mit dem Spiel begonnen, und der Client übermittelt
wiederholt die vermutete Farbfolge, bis der Server entweder kommuniziert, dass
die Folge der Geheimfolge entspricht (Antwortfeld rot ist 5), oder einen
Fehler übermittelt. Am Ende des Spiels werden entweder die Anzahl der
gespielten Runden oder der zuletzt übermittelte Fehler ausgegeben. Danach soll
der Socket geschlossen und das Programm beendet werden (Rückgabewert siehe
nächster Absatz).

\begin{verbatim}
Client: SYNOPSIS
  client <server-hostname> <server-port>
        EXAMPLE
   client localhost 1280
\end{verbatim}

% Verbinden Sie den Socket mittels \emph{connect}. Senden Sie mittels \emph{write} die Anfragen an de Server, und empfangen Sie de Antwort des Servers mittels \emph{read} und schicken ihn mittels \emph{write} wieder zurück. 
\paragraph{Fehlermeldungen und Rückgabewerte:} Ist eines der beiden
Fehlerstatusbits (siehe Abschnitt~\nameref{sec:prot}) gesetzt, sollen sowohl der
Client als auch der Server terminieren. Bei einem Parit\"atsfehler (Bit $6$
der Serverantwort gesetzt) soll die Meldung ``Parity error'' ausgegeben
werden (Rückgabewert 2). Bei einer Überschreitung der maximalen
Rundenanzahl, wenn also der Spieler die Folge nicht erraten konnte (Bit 7
gesetzt), geben Sie die Meldung ``Game lost'' aus (Rückgabewert 3). Sind beide Bits gesetzt, sind beide Meldungen auszugegeben und das
Programm mit Rückgabewert 4 zu terminieren. Bei sonstigen Fehlern (z.B.
ungültige Kommandozeilenargumente, Verbindungsfehler, \ldots) soll eine
informative Fehlermeldung ausgegeben und mit Rückgabewert EXIT\_FAILURE (1)
terminiert werden. Alle Fehlermeldungen müssen auf \emph{stderr} ausgegeben
und von einem Zeilenumbruch gefolgt werden.

Endet das Spiel regulär (der Spieler errät die geheime Farbfolge), so soll die
Anzahl gespielter Runden auf \texttt{stdout} ausgegeben werden (Rückgabewert
0).

\subsection*{Protokoll}
\label{sec:prot}
\emph{Server} und \emph{Client} kommunizieren in Runden wie nachfolgend
gezeigt. Der Client \"ubermittelt pro Runde genau zwei Bytes, der Server genau
ein Byte. Dabei muss im Falle des Clients stets zuerst das niedrigerwertige
Byte (Bits 0-7), und dann das höherwertige Byte (Bits 8-15) übertragen werden,
unabhängig von der Architektur der Zielplattform.

\subsubsection*{Client}
Der \emph{Client} schickt an den Server Nachrichten im folgenden Format:\\

\begin{bytefield}[boxformatting={\centering\itshape},bitwidth=1.5em]{16}
   \bitheader[b]{0-15} \\
   \bitbox{1}{p} & \bitbox{3}{color R} & \bitbox{3}{color} & \bitbox{3}{color} & \bitbox{3}{color} & \bitbox{3}{color L} 
\end{bytefield}

Der Wert \verb|color L| entspricht der Farbe ganz links am Spielbrett, der Wert \verb|color_R| der Farbe ganz
rechts. Den Farben werden die Werte entsprechend dem folgenden \verb|enum| zugeordnet: \verb|enum color {beige = 0, darkblue, green, orange, red, black, violet, white}|.

Der Wert \verb|p| entspricht einem \emph{Parity Bit} \"{u}ber die einzelnen Bits der Farben (Bit~0 bis Bit~14). Um dieses zu Berechnen, werden die einzelnen Bits mit einer \emph{xor}-Operation verkn\"{u}pft.

\subsubsection*{Server}

Der \emph{Server} empf\"{a}ngt die Nachricht des Clients und wertet den aktuellen Versuch des Clients aus. Der Wert
\verb|number red| entspricht der Anzahl an richtig erratenen Farben an der richtigen Position, \verb|number white| der
Anzahl an richtig erratenen Farben an der falschen Position (siehe Abschnitt~\nameref{sec:aufgabenstellung}). Das Feld \verb|status| hat
folgende Bedeutung: Ist Bit $6$ gesetzt, wurde das \emph{Parity Bit} vom Client nicht richtig berechnet. Ein gesetztes
Bit an Position 7 bedeutet dass die maximale Anzahl an Versuchen \"{u}berschritten wurde (Sie haben verloren). \\

\begin{bytefield}[boxformatting={\centering\itshape},bitwidth=2.2em]{8}
   %\settowidth{\bitwidth}{~number white~}
   \bitheader[b]{0-7} \\
   \bitbox{2}{status} & \bitbox{3}{number white} & \bitbox{3}{number red}
\end{bytefield}


\subsection*{Bewertung}
\label{sec:grading}
Um das Beispiel erfolgreich zu lösen, müssen sowohl die Implementierung des
Servers als auch des Clients korrekt funktionieren (siehe allgemeine Beispielanforderungen) und folgenden Anforderungen genügen:

\paragraph{Server:}
Teile des Servers sind bereits vorgegeben. Bitte verwenden Sie dieses Template und erweitern Sie es entsprechend.
Der Server muss die korrekte Antwort auf gültige Anfragen
übermitteln. Entspricht bei der 35ten Anfrage die vermutete Folge nicht der
Geheimfolge, so muss das Bit 7 der Serverantwort gesetzt werden. Genau dann
wenn das Paritätsbit falsch berechnet wurde, ist das Bit 6 der Antwort zu
setzten. Schließlich muss der Server korrekt auf die in Abschnitt
\nameref{sec:implhints} beschriebenen Signale reagieren.

\paragraph{Client:} Der Client muss jeweils innerhalb von einer Sekunde (auf
dem Abgaberechner im TI Labor) eine Anfrage an den Server generieren. 
% Weiters muss jedes Spiel innerhalb von 35 Runden gewonnen werden.
Sendet der Server einen Fehlercode, so muss der Client sofort mit der
entsprechenden Fehlermeldung terminieren.

\paragraph{Bonuspunkte:} Für gute und ausgezeichnete L\"osungen werden Bonuspunkte vergeben.
Die Qualität einer Lösung wird dadurch bestimmt, wie viele Runden im Schnitt
benötigt werden, um ein Spiel zu gewinnen (average rounds per game). 
Ben\"otigt Ihre L\"osung im Durchschnitt weniger als 20 Runden, und verliert kein
einziges Spiel, so erhalten Sie 5 Bonuspunkte. Weitere 5 Bonuspunkte werden vergeben,
wenn der Client das Spiel in durchschnittlich 8 oder weniger Runden gewinnt.
\end{document}

