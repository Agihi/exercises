\input{../../../template.ltx}

\begin{document}
\osuetitle{1}

\section*{Aufgabenstellung B -- Coffee Maker}\label{sec:aufgabenstellung}

Am Institut für Technische Informatik wurde eine Kaffeemaschine mit einem
Raspberry Pi mit Netzwerkanbindung und automatischer Kaffeekapsel-Versorgung
erweitert. Nun soll ein Service ``Kaffeeproduktion'' entwickelt werden.

\subsection*{Anleitung}
Implementieren Sie einen Client und einen Server, die mittels TCP/IP
miteinander kommunizieren. Angestellte des Instituts starten den Client, um
eine Anfrage an die Kaffeemaschine zu senden. Der Server soll auf der
Kaffeemaschine Anfragen bearbeiten und die Kaffeeproduktion einleiten.

Nachdem eine Verbindung zwischen dem Client und dem Server hergestellt wurde,
wird die Anfrage übermittelt. Angestellte können Tassengröße und
Geschmacksrichtung des Kaffees wählen. Die Kaffeemaschine soll mit
Produktionsdauer oder Status antworten.

Verwenden Sie geeignete Ausgaben (auf \osueglvar{stdout}) am Server und Client
um Anfragen und Kaffeeproduktion folgen zu können. Alle Fehlermeldungen müssen
auf \osueglvar{stderr} ausgegeben.

\subsubsection*{Server}
Der Server soll auf eingehende Verbindungen warten. Sobald eine Verbindung
akzeptiert wurde, wird die Anfrage bearbeitet. Neue Verbindungen werden
unmittelbar nach Bearbeitung der Anfrage (d.h. lesen, berechnen, antworten)
wieder zugelassen, d.h. der Server wartet nicht auf die Beendigung der
Kaffeeproduktion (welche hier selbst nicht implementiert wird). Der Server
sollte sich aber den Zeitstempel der Fertigstellung merken, um Anfragen während
einer bereits laufenden Kaffeeproduktion korrekt bearbeiten zu können
(Produktionsdauer).

Der Server verwaltet den aktuellen Status der Kaffeemaschine, welcher nach
jeder Anfrage aktualisiert werden muss. Initial soll der Wassertank voll und
die Kapselkammer, die gebrauchte Kapseln sammelt, leer sein. Pro Anfrage: soll
je nach Tassengröße der Inhalt des Wassertanks verringert werden und die Anzahl
der Kapseln in der Kapselkammer inkrementiert werden. Ist das Wasser für die
aktuelle Anfrage nicht mehr ausreichend oder die Kapselkammer voll, soll ein
Fehlercode an den Client zurück gesendet werden. Kann der Kaffee produziert
werden, wird die Dauer der Produktion in Sekunden zurückgeschickt.

Sobald der Server eines der Signale \osueconst{SIGINT} oder
\osueconst{SIGTERM} empfängt, soll der Socket geschlossen und das
Programm mit Rückgabewert 0 beendet werden.

\begin{verbatim}
SYNOPSIS
    server [-p PORT] [-l WATER] [-c CUPS]
\end{verbatim}

Die Größe des Wassertanks (default: 1l) und der Kapselkammer (default: 10
Kapseln) kann beim Starten des Servers ausgewählt werden. Die Portnummer ist
per default 1821.

\subsubsection*{Client}
Legen Sie zuerst einen TCP/IP-Socket an. Stellen Sie dann die zum Hostnamen des
Servers zugehörige IP-Adresse fest und verbinden Sie sich mit dem
Server. Unmittelbar nach Verbindungsaufbau wird die Anfrage zur
Kaffeeproduktion übermittelt und die Antwort des Servers ausgewertet und
angezeigt. Danach soll der Socket geschlossen und das Programm beendet werden.

Der Client sollte bei erfolgreichem Verbindungsaufbau mit dem Server, sofort
eine Antwort bekommen. Tritt ein Fehler auf (z.B.: keine Verbindung möglich),
beendet der Client mit einem Fehlercode. Der Client sollte nie blockieren, Sie
können daher auf eine Signalbehandlung im Client verzichten.

\begin{verbatim}
SYNOPSIS
    client [-h HOSTNAME] [-p PORT] SIZE FLAVOUR
\end{verbatim}

Dem Client können Hostname (default: \emph{localhost}) und Portnummer (default:
\emph{1821}) übergeben werden. Die Tassengröße (\osueconst{SIZE}, ganzzahlig)
und Geschmacksrichtung (\osueconst{FLAVOUR}, C-String) des Kaffees sollen über
die Argumente gesetzt werden.

\subsection*{Protokoll}
\label{sec:prot}
Der Client übermittelt Tassengröße und Geschmacksrichtung. Folgende Werte
sollen unterstützt werden:
%
\begin{description}
\item[Tassengröße:] 0 - 330 ml
\item[Geschmacksrichtung:] \{Decaffeinato, Kazaar, Volluto, Ciocattino,
  Vanilio, \dots\} - es sollen bis zu 32
  Geschmacksrichtungen\footnote{\url{https://www.nespresso.com/at/de/grands-crus-uebersicht}}
  unterstützt werden.
\end{description}

Die Anfrage soll in 2 Bytes übermittelt werden. Das letzte Bit soll ein Parity
Bit\footnote{\url{https://en.wikipedia.org/wiki/Parity_bit}}
implementieren. Dieses ist am Server zu überprüfen.

Die Antwort des Servers ist vom Status der Kaffeemaschine (Wassertank,
Kapselkammer) und der in Produktion befindlichen Kaffees abhängig. Je nach
Status ergeben sich folgende Inhalte der Antwort-Nachricht.
%
\begin{description}
\item[Status:] \{OK - Kaffee kann produziert werden, NOK - Kaffeemaschine muss
  gewartet werden\}
  \begin{description}
  \item[OK $\rightarrow$ Dauer der Kaffeeproduktion:] Dauer in Sekunden, die
    durch die Tassengröße (Wassermenge) bestimmt wird und Restdauer von
    Kaffeeproduktion. Pro 10ml wird 1s benötigt. Sollte die Kaffeeproduktion
    noch laufen, muss die Restdauer dazuaddiert werden. Runden Sie auf
    ganzzahlige Sekunden-Werte.
  \item[NOK $\rightarrow$ Fehlercode:] \{Wassertank leer, Kapselkammer voll\}
  \end{description}
\end{description}

Die Antwort soll in 1 Byte übermittelt werden. Auch die Antwort soll ein Parity
Bit enthalten und vom Client überprüft werden.

Definieren Sie ein geeignetes Format der Nachrichten (z.B.: IDs für die
Geschmacksrichtungen vergeben). Stellen Sie sicher, dass die Übertragung
unabhängig von der Architektur funktioniert.


%% \osueadvertise{Wollen Sie den Client erweitern und auf das Robot Operating
%%   System\footnote{\url{http://www.ros.org/}} portieren?}

\osueguidelinesone

\end{document}
