% dsort, 11/2010, Thomas Ogrisegg
\input{../../template.ltx}

\begin{document}

\osuetitle{2}

\section*{Aufgabenstellung}

\begin{verbatim}
    SYNOPSIS:
        dsort "command1" "command2"
\end{verbatim}

Schreiben Sie ein Programm, das die beiden Kommandos \osuecmd{command1}
und \osuecmd{command2} ausführt, deren Ausgaben einliest und in ein
gemeinsames Array speichert. Dieses Array wird dann sortiert und an
das Unix-Kommando \osuecmd{uniq -d} weitergegeben. Die Ausgabe Ihres
Programmes soll also identisch sein mit jener des folgenden
Shellskripts\footnote{Anmerkung: \$1 und \$2 stehen für command1 und
command2}:

\begin{verbatim}
#!/bin/sh

( $1; $2 ) | sort | uniq -d
\end{verbatim}

\subsection*{Anleitung}

Das Programm soll für die beiden Kommandos jeweils mittels Pipes die
Ausgabe der Kindprozesse zeilenweise einlesen und in ein Array
speichern. Die Kindprozesse werden mit \osuefunc{fork(2)} erzeugt und
sollen die Bourne-Shell mit dem Parameter \osuearg{-c} ausführen (also
\osuecmd{/bin/sh -c}). Dadurch ist es möglich dem Programm mittels
Anführungszeichen ein Shellkommando zu übergeben -- die Shell kümmert
sich um alles Weitere.

Das vom Programm erzeugte Array mit den Ausgaben der Kommandos soll
dann sortiert werden. Danach wird ein neuer Kindprozess gestartet,
welcher das Programm \osuecmd{uniq} mit der Option \osuearg{-d} ausführen
soll. Das sortierte Array wird diesem Prozess über die Standardeingabe
übergeben (wieder \osuefunc{pipe(2)} verwenden), indem zeilenweise in die
Pipe geschrieben wird.

Der \osuecmd{uniq}-Prozess sucht im \osuearg{-d}-Modus alle mehrmals
nacheinander vorkommenden Zeilen heraus und gibt sie auf die
Standardausgabe aus. \osuecmd{uniq} ist ein
\textsc{Unix}-Standardprogramm und muss von Ihnen nicht programmiert
werden.

Beispiel:
\begin{verbatim}
  $ ./dsort "cat /etc/passwd" "cat /etc/passwd.tmp"
\end{verbatim}

Dieser Befehl würde die beiden Dateien /etc/passwd und /etc/passwd.tmp
vergleichen und alle Zeilen, die in beiden Dateien (oder in einer Datei
mehrmals) vorhanden sind, auf die Standardausgabe ausgeben.

Beispiel:
\begin{verbatim}
  $ ./dsort "seq 0 3 100" "seq 0 11 100"
\end{verbatim}

Dieser Befehl gibt alle gemeinsamen Teiler von 3 und 11, die zwischen
0 und 100 liegen, aus.

\subsection*{Hinweise}

Beachten Sie insbesonders folgende Punkte:

\begin{itemize}
\item Sie müssen sowohl das Array als auch die Strings des Arrays
dynamisch allozieren und auf eine saubere Freigabe achten. Hierfür
stehen die Funktionen \osuefunc{malloc(3)}, \osuefunc{realloc(3)},
\osuefunc{free(3)} und bspw.\ auch \osuefunc{strdup(3)} zur Verfügung.
\item Zum Sortieren der Zeilen können Sie \osuefunc{qsort(3)} verwenden.
\item Ihr Programm sollte zeilenweise einlesen (z.B.\ mit
\osuefunc{fgets(3)}) bzw.\ sortieren.
\item Sie können davon ausgehen, dass eine Eingabezeile nicht länger
als 1023 echte Zeichen ist.
\item Die Funktion \osuefunc{popen(3)} darf nicht verwendet werden.
\item Achten Sie darauf, dass der Elternprozess sich erst beendet,
nachdem alle Kindprozesse terminiert haben.
\item Gewisse Nicht-ANSI-Funktionen (wie etwa \osuefunc{strdup(3)}) werden
erst sichtbar, wenn mit\\\osuearg{-D\_XOPEN\_SOURCE=500} kompiliert wird.
\item Achten Sie unbedingt darauf, dass alle Ressourcen auch wieder
freigegeben werden (besonders im Fehlerfall)!
\end{itemize}

\osueguidelinestwo

\end{document}
