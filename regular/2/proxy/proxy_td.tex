% TeX source file
% Sysprog SS 2005
% Beispiel 3: proxy
% Christian Steiner


\documentclass{article}
\usepackage[german]{babel}
\usepackage[T1]{fontenc}
\usepackage[utf8]{inputenc}
\usepackage{a4wide}
\usepackage{url}
\parindent0pt
\addtolength{\parskip}{10pt}

\begin{document}

\begin{center}
\begin{Large}
OPERATING SYSTEMS BEISPIEL 2
\end{Large}
\end{center}

\section*{Aufgabenstellung}

Implementieren Sie ein Programm, welches ein auf der Kommandozeile angegebenes
Programm in einem Kindprozess ausführt und je nach angegebenen Optionen 
\emph{stdin}, \emph{stdout} und \emph{stderr} in files umlenkt.

\begin{verbatim}
    SYNOPSIS:
       proxy [-i infile] [-o outfile] [-e errfile] <cmd> [options]
\end{verbatim}

Wird ein Ziel nicht angegeben (eine oder mehrere der Optionen \verb_-i_,
\verb_-o_ oder \verb_-e_ fehlt) oder wurde \verb_-_ als Filename angegeben, soll
der entsprechende Ein-/Ausgabekanal unverändert bleiben. Der Rückgabewert des
\verb_proxy_-Programms soll dem Rückgabewert des ausgeführten Programms
entsprechen.

\section*{Anleitung}

Verwenden Sie \emph{getopt(3)} zur Argumentbehandlung. Benötigt das
auszuführende Programm eine oder mehrere Optionen, können Sie durch Voranstellen
von \verb|--| eine Fehlermeldung von {\em getopt} vermeiden – das Parsen der
Parameter endet dann an dieser Stelle.\\
(Beispiel: \verb|./proxy -o outfile -- ls -o *|)

Erzeugen Sie anschließend mit \emph{fork(2)} einen Kindprozess. Lenken Sie
mittels \emph{dup(2)} oder \emph{dup2(2)} die entsprechenden Filedeskriptoren um
(vergessen Sie nicht, den ursprünglichen Deskriptor gegebenenfalls vorher zu
schließen) und führen Sie zuletzt das Programm samt Optionen mithilfe von
\emph{execvp(2)} aus. Vergessen Sie im Vaterprozess nicht, den Exitstatus des
Kindprozesses abzuholen (und so die Entstehung eines Zombieprozesses zu
vermeiden).

\section*{Testen}

Verwenden Sie \verb_proxy_, um mehrere Programme auszuführen:

\begin{verbatim}
	./proxy cat
	
	./proxy -i Makefile -o Makefile.cat cat

	./proxy -i - -o - -- grep -ni
\end{verbatim}

\section*{Richtlinien}
Bitte beachten Sie auch die Allgemeinen Hinweise zur Beispielgruppe 2 und die Richtlinien f\"ur die Erstellung von C-Programmen auf der \"Ubungswebsite.
Insbesondere ist es ab dieser Beispielgruppe notwendig, die Dokumentation in Doxygen zu f\"uhren. Es muss zumindest das HTML Output generierbar sein. Bitte dokumentieren Sie ausnahmslos alle Funktionen (auch static-Funktionen, siehe \verb|EXTRACT_STATIC| im Doxygen Cfg-File). Eine kurze Einf\"uhrung haben wir Ihnen auf: \url{http://wiki.vmars.tuwien.ac.at/index.php/Doxygen_Primer} bereitgestellt. Achten Sie weiters darauf, dass nach au{\ss}en hin sichtbare Funktionen (exportierte Funktionen) im Header File beschrieben werden und lokale (static Funktionen) nur im C File. Sie sollten auch Ihre Typen (insb. structs), Konstanten und globale Variablen dokumentieren. 
\end{document}
