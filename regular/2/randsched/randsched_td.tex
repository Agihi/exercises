\documentclass{article}
\usepackage[german]{babel}
\usepackage[T1]{fontenc}
\usepackage[utf8]{inputenc}
\usepackage{a4wide}
\usepackage{url}
\parindent0pt
\addtolength{\parskip}{10pt}

\begin{document}
\begin{center}
\begin{Large}
OPERATING SYSTEMS BEISPIEL 2
\end{Large}
\end{center}
\section*{Aufgabenstellung}

Sie sind der neue Systemadministrator im lokalen Atomkraftwerk. Schon bald ist
Ihnen klar, dass im Wesentlichen einfach nur alle paar Minuten ein Programm
aufgerufen werden muss, dessen Output in eine Protokolldatei abzutippen ist.
Falls doch mal ein Fehler auftritt, muss aber von Ihnen der sofortige
Reaktor-Shutdown eingeleitet werden. Sich Ihrer kritischen, dennoch gefährlichen
und schlecht bezahlten Position bewusst, fassen Sie schnell den Entschluss,
diesen Vorgang möglichst transparent zu automatisieren.

\subsection*{Anleitung}
Implementieren Sie die folgenden Hilfsprogramme, um den Reaktor zu simulieren:

\begin{verbatim}
          SYNOPSIS:  rventgas

          SYNOPSIS:  rshutdown
\end{verbatim}

Das Programm \emph{rventgas} soll mit einer Wahrscheinlichkeit von 6/7
erfolgreich mit der Meldung \verb_STATUS OK_ terminieren, sonst soll
\verb_PRESSURE TOO HIGH - IMMEDIATE SHUTDOWN REQUIRED_ ausgegeben werden und die
Terminierung erfolgt mit einem Fehlercode.

Das Programm \emph{rshutdown} schreibt mit einer Wahrscheinlichkeit von 1/3 die
Meldung\\
\verb_SHUTDOWN COMPLETED_ auf \emph{stdout} und terminiert, sonst soll
\verb_KaBOOM!_ ausgegeben und ein Fehlercode zurückgegeben werden.

Nun zum eigentlichen Teil der Aufgabe: Implementieren Sie das folgende Programm:

\begin{verbatim}
SYNOPSIS:  schedule [-s <seconds>] [-f <seconds>] <program> <emergency> <logfile>

-s <seconds>   Zeitfenster Anfang (Default: 1 Sekunde)
-f <seconds>   Zeitfenster Ende (Default: 0 Sekunden)
<program>      Programm inkl. Pfad, das wiederholt ausgefuehrt werden soll
<emergency>    Programm inkl. Pfad, das im Fehlerfall ausgefuehrt wird
<logfile>      Pfad zu einer Datei, in der die Ausgabe von <program> sowie
               Erfolg/Misserfolg von <emergency> protokolliert werden
\end{verbatim}

Das Programm \emph{schedule} erzeugt einen Kindprozess, der wiederholt zufällig
irgendwann im spezifizierten Zeitfenster (in $s$ bis $s+f$ Sekunden, als
Granularität genügt eine volle Sekunde) das Programm \emph{program} als
Enkelkind startet und auf seine Beendigung wartet. Falls \emph{program} mit
einem Fehlercode terminiert, wird einmalig \emph{emergency} ausgeführt und der
Kindprozess beendet. Wird z.\,B.\ \emph{schedule} mit \verb_-s 5_ und
\verb_-f 2_ aufgerufen, wartet der Kindprozess zunächst zwischen 5 und 7
Sekunden und führt dann \emph{program} aus. Terminiert \emph{program}
fehlerfrei, wartet der Kindprozess wieder zwischen 5 und 7 Sekunden, bevor er
\emph{program} ein weiteres Mal ausführt.

Weiters soll der Kindprozess alle Daten, die der \emph{program}-Enkelprozess auf
\emph{stdout} ausgegeben hat, über eine Pipe an den Vaterprozess schicken.
Dieser gibt das Gelesene auf (sein) \emph{stdout} aus und fügt es gleichzeitig
an die angegebene \emph{logfile} an. Ausgaben von \emph{emergency} müssen nicht
im Logfile aufscheinen; es genügt zu protokollieren, ob \emph{emergency} ohne
Fehlercode terminiert hat. Nachdem \emph{emergency} beendet ist, soll auch
\emph{schedule} terminieren. Implementieren Sie außerdem noch eine
Signalbehandlung, um \emph{schedule} sauber beenden zu können, ohne auf eine
Fehlfunktion des Reaktors hoffen zu müssen.

\subsection*{Testen}

Das typische Output sollte so ähnlich aussehen:
\begin{verbatim}
$# ./schedule -s 2 -f 5 ./rventgas ./rshutdown logfile
STATUS OK
STATUS OK
STATUS OK
STATUS OK
STATUS OK
PRESSURE TOO HIGH - IMMEDIATE SHUTDOWN REQUIRED
SHUTDOWN COMPLETED.
EMERGENCY SUCCESSFUL

$# cat logfile
STATUS OK
STATUS OK
STATUS OK
STATUS OK
STATUS OK
PRESSURE TOO HIGH - IMMEDIATE SHUTDOWN REQUIRED
EMERGENCY SUCCESSFUL
\end{verbatim}
Die Meldung \verb_SHUTDOWN COMPLETED_ ist eigentlich von \emph{rshutdown} und
scheint nicht im logfile auf, wo aber sehr wohl \verb_EMERGENCY SUCCESSFUL_
anzeigt, dass der \emph{emergency}-Prozess erfolgreich war.

\section*{Richtlinien}
Bitte beachten Sie auch die Allgemeinen Hinweise zur Beispielgruppe 2 und die Richtlinien f\"ur die Erstellung von C-Programmen auf der \"Ubungswebsite.
Insbesondere ist es ab dieser Beispielgruppe notwendig, die Dokumentation in Doxygen zu f\"uhren. Es muss zumindest das HTML Output generierbar sein. Bitte dokumentieren Sie ausnahmslos alle Funktionen (auch static-Funktionen, siehe \verb|EXTRACT_STATIC| im Doxygen Cfg-File). Eine kurze Einf\"uhrung haben wir Ihnen auf: \url{http://wiki.vmars.tuwien.ac.at/index.php/Doxygen_Primer} bereitgestellt. Achten Sie weiters darauf, dass nach au{\ss}en hin sichtbare Funktionen (exportierte Funktionen) im Header File beschrieben werden und lokale (static Funktionen) nur im C File. Sie sollten auch Ihre Typen (insb. structs), Konstanten und globale Variablen dokumentieren. 
\end{document}
