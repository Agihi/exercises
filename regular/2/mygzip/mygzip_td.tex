\input{../../template.ltx}

\begin{document}

\osuetitle{2}

\section*{Aufgabenstellung}
Schreiben Sie ein Programm, das Eingaben mit \emph{gzip(1)} komprimiert.
\begin{verbatim}
    SYNOPSIS
        mygzip [file]
\end{verbatim}


\subsection*{Anleitung}
Das Programm erstellt zwei Pipes und führt zwei mal \emph{fork(2)} aus (und erzeugt damit zwei Kinder – nicht Kind und Enkelkind).
Das erste Kind biegt \emph{stdin} auf die erste und \emph{stdout} auf die zweite Pipe um und startet mit \emph{execlp(3)} das Programm \emph{gzip(1)} mit dem Parametern \emph{-cf}.
Der Vater liest die zu komprimierenden Daten von \emph{stdin} ein und schreibt sie über die erste Pipe zum gzip-Prozess.
Das zweite Kind liest über die zweite Pipe vom \emph{gzip}-Prozess und schreibt die gelesenen Daten in die Datei \emph{file} bzw.\ auf \emph{stdout} wenn keine Datei angegeben wurde.
Achten Sie darauf, das Sie die Dateien binär öffnen.

\osueguidelinestwo

\end{document}
