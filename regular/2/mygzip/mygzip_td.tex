\input{../../template.ltx}

\begin{document}

\osuetitle{2}

\section*{Aufgabenstellung -- mygzip}

Schreiben Sie ein Programm, das Eingaben mit \osueprog{gzip(1)} komprimiert.

\begin{verbatim}
    SYNOPSIS
        mygzip [file]
\end{verbatim}

\subsection*{Anleitung}

Das Programm erstellt zwei Pipes und führt zwei mal \osuefunc{fork(2)} aus (und
erzeugt damit zwei Kinder -- nicht Kind und Enkelkind). Das erste Kind biegt
\osueglvar{stdin} auf die erste und \osueglvar{stdout} auf die zweite Pipe um
und startet mit \osuefunc{execlp(3)} das Programm \osueprog{gzip(1)} mit dem
Parametern \osuearg{-cf}. Der Vater liest die zu komprimierenden Daten von
\osueglvar{stdin} ein und schreibt sie über die erste Pipe zum
\osueprog{gzip}-Prozess. Das zweite Kind liest über die zweite Pipe vom
\osueprog{gzip}-Prozess und schreibt die gelesenen Daten in die Datei
\osuefilename{file} bzw.\ auf \osueglvar{stdout}, wenn keine Datei angegeben
wurde.

Achten Sie darauf, das Sie die Dateien binär öffnen.

\osueguidelinestwo

\end{document}
