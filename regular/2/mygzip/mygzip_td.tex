\documentclass{article}
\usepackage[german]{babel}
\usepackage[T1]{fontenc}
\usepackage[utf8]{inputenc}
\usepackage{url}
\usepackage{a4wide}
\parindent0pt
\addtolength{\parskip}{10pt}

\begin{document}

\begin{center}
\begin{Large}
OPERATING SYSTEMS BEISPIEL 2
\end{Large}
\end{center}

\section{Benutzung von \emph{gzip}}

\subsection{Aufgabenstellung}
Schreiben Sie ein Programm, das Eingaben mit \emph{gzip(1)} komprimiert.
\begin{verbatim}
    SYNOPSIS
        mygzip [file]
\end{verbatim}


\subsection{Anleitung}  
Das Programm erstellt zwei Pipes und führt zwei mal \emph{fork(2)} aus (und erzeugt damit zwei Kinder – nicht Kind und Enkelkind).
Das erste Kind biegt \emph{stdin} auf die erste und \emph{stdout} auf die zweite Pipe um und startet mit \emph{execlp(3)} das Programm \emph{gzip(1)} mit dem Parametern \emph{-cf}.
Der Vater liest die zu komprimierenden Daten von \emph{stdin} ein und schreibt sie über die erste Pipe zum gzip-Prozess.
Das zweite Kind liest über die zweite Pipe vom \emph{gzip}-Prozess und schreibt die gelesenen Daten in die Datei \emph{file} bzw.\ auf \emph{stdout} wenn keine Datei angegeben wurde.
Achten Sie darauf, das Sie die Dateien binär öffnen.

\section*{Richtlinien}
Bitte beachten Sie auch die Allgemeinen Hinweise zur Beispielgruppe 2 und die Richtlinien f\"ur die Erstellung von C-Programmen auf der \"Ubungswebsite.
Insbesondere ist es ab dieser Beispielgruppe notwendig, die Dokumentation in Doxygen zu f\"uhren. Es muss zumindest das HTML Output generierbar sein. Bitte dokumentieren Sie ausnahmslos alle Funktionen (auch static-Funktionen, siehe \verb|EXTRACT_STATIC| im Doxygen Cfg-File). Eine kurze Einf\"uhrung haben wir Ihnen auf: \url{http://wiki.vmars.tuwien.ac.at/index.php/Doxygen_Primer} bereitgestellt. Achten Sie weiters darauf, dass nach au{\ss}en hin sichtbare Funktionen (exportierte Funktionen) im Header File beschrieben werden und lokale (static Funktionen) nur im C File. Sie sollten auch Ihre Typen (insb. structs), Konstanten und globale Variablen dokumentieren. 
\end{document}
