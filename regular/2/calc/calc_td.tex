% TeX source file
% Sysprog SS 2005
% Beispiel 3: calculator
% Martin Kirner
\input{../../template.ltx}

\begin{document}

\osuetitle{2}

\section*{Aufgabenstellung}

\begin{verbatim}
SYNOPSIS:
        calculator
        $> <zahl1> <zahl2> <operator>

BNF:
        <zahl>     ::= -?[0-9]+
        <operator> ::= +|-|*|/

\end{verbatim}

Schreiben Sie einen Taschenrechner als Vaterprozess, der seinen Input
von \emph{stdin} als String einliest und den String an einen
Kindprozess weiterleitet. Dieser extrahiert aus dem String die
Operanden und die Operation, berechnet das Ergebnis und gibt dieses
als String wieder zurück. Das Ergebnis wird vom Vaterprozess
anschließend am Bildschirm wieder ausgegeben. Da als Zahlen
Integerwerte (maximal 65535) angenommen werden sollen, kann die
Eingabezeichenkette auf 15 Zeichen beschränkt werden.


Beispielsweise soll die Eingabe \verb_10 15 +_ das Ergebnis \verb_25_
liefern.

\section*{Anleitung}

Das Programm soll nach dem Starten in einer Schleife wiederholt
Rechnungen von der Tastatur (\emph{stdin}) einlesen. Zur
Berechnung soll Ihr Programm einen Kindprozess erzeugen, an den
es die Eingabe über eine Pipe weiterleitet. Dieser Kindprozess
leitet nach der Umwandlung und Berechnung das Ergebnis über eine
zweite Pipe an den Vaterprozess zurück. Der Vaterprozess
gibt dieses Ergebnis wieder am Bildschirm (\emph{stdout}) aus.


Dieser Vorgang soll solange wiederholt werden, bis der Vaterprozess
EOF (Ctrl-D von der Tastatur) liest. In diesem Fall ist die Pipe zum
Kindprozess zu schließen. Der Kindprozess erhält dadurch beim Lesen
ebenfalls EOF und terminiert. Der Vaterprozess soll auf die
Terminierung des Kindprozesses warten, alle benötigten Ressourcen
(Pipes) an das System zurückgeben (schließen) und dann ebenfalls
terminieren.


Sie können davon ausgehen, dass alle die Rechnungen richtig eingegeben
werden – es ist also keine Syntaxüberprüfung notwendig. Auch können
Sie die Ergebnisse abrunden (Kommastellen abschneiden) – die Eingabe
\verb_5 2 /_ ergibt als Ergebnis einfach \verb_2_ anstelle von
\verb_2.5_.

\osueguidelinestwo

\end{document}
