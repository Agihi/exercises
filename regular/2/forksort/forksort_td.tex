\input{../../template.ltx}

\begin{document}

\osuetitle{2}

\section*{Aufgabenstellung -- forksort}
Schreiben Sie ein Programm, das Eingaben sortiert.
\begin{verbatim}
    SYNOPSIS
        forksort
\end{verbatim}

\subsection*{Anleitung}
Das Programm soll die Daten von \osueglvar{stdin} lesen. Die erste Zeile muss
eine Zahl größer Null sein, die die Anzahl der zu sortierenden Strings angibt.
Die darauffolgenden Zeilen enthalten die Strings selbst; Sie können die
Strings auf je 62 echte Zeichen begrenzen. (Vergessen Sie nicht dafür eine
Konstante zu definieren!)

\subsection*{Beispiele}
\begin{verbatim}
$ ./forksort
5
Heinrich
Anton
Theodor
Dora
Hugo
Anton
Dora
Heinrich
Hugo
Theodor
\end{verbatim}

\subsection*{Arbeitsweise}
Forksort funktioniert wie eine rekursive Merge-Sort-Variante\footnote{Siehe insb. die graphisch aufbereitete Beispielausf{\"u}hrung auf: http://de.wikipedia.org/wiki/Mergesort}.
\begin{enumerate}
\item Lesen Sie die Anzahl der Strings ein.
\item Ist die Anzahl gleich Eins, dann geben Sie das "`sortierte"' Ergebnis
  aus (d.h. den Eingabe-String selbst).
\item Ist die Anzahl größer Eins, erstellen Sie zuerst vier Unnamed-Pipes (zwei
  je Kind) und erzeugen dann zwei Kindprozesse (nicht Kind- und Enkelkind) mittels
  \osuefunc{fork(2)}. Verwenden Sie \osuefunc{dup2(2)}
  und \osuefunc{execlp(3)} um \osueglvar{stdin}/\osueglvar{stdout} der neuen
  Prozesse auf eigene Streams umzuleiten und danach Forksort rekursiv
  aufzurufen.
\item Schreiben Sie je die Hälfte der Strings auf \osueglvar{stdin} der
  beiden Kinder im zuvor spezifiziertem Eingabeformat (d.h., Anzahl der Strings,
  Newline, die Eingabestrings zeilenweise).
  Ist die Anzahl der Strings ungerade, dann wählen Sie
  ein beliebiges Kind aus, das einen String mehr bekommt.
  %Vergessen Sie nicht
  %auch die Anzahl der Strings an das jeweilige Kind zu übermitteln.
\item Lesen Sie nun die Ausgabe der Kinder zeilenweise ein, und
  geben Sie diese aber nun ''verschmolzen'', d.h. sortiert zeilenweise aus
  (dieser Vorgang nennt sich \emph{Merge}).
  Beachten Sie, dass die Ausgabe der Kinder bereits fuer
  sich sortiert ist.  Sie können also
  solange die Strings des ersten Kindes ausgeben, bis der kleinste String des
  zweiten Kindes kleiner als der kleinste, noch nicht ausgegebene, String des
  ersten Kindes ist. Danach lesen Sie vom zweiten Kind, bis der String des
  ersten Kindes wiederum kleiner ist. Wiederholen Sie diesen Vorgang solange,
  bis alle Strings der Kinder abgearbeitet sind.

  Der Merge-Vorgang muss einen linearen Aufwand haben.
\item Verwenden Sie \osuefunc{wait(2)} oder \osuefunc{waitpid(2)}, um den
  Exit-Status der beiden Kinder zu lesen.
\end{enumerate}

\subsection*{Hinweise}
Achten Sie auf korrekte Terminierungsbedingungen von forksort um ''Endlos-Rekursionen'' zu vermeiden\footnote{http://en.wikipedia.org/wiki/Fork\_bomb}.
Dazu sind zwei Regeln zu beachten:
\begin{enumerate}
\item Forken Sie nur, wenn die Anzahl der zu sortierenden Strings größer eins ist.
\item Die an das jeweilige Kind übergebene Anzahl muss stets kleiner sein als
  die des Elternprozesses.
\end{enumerate}
Beginnen Sie am besten, indem Sie eine Variante schreiben, die nur einen String
sortieren kann. Erweitern Sie nun Ihr Programm für zwei Strings, indem Sie den beschriebenen
Fork-Vorgang implementieren und jeweils einen String an jedes der beiden Kinder schreiben. Diese
können ja schon einen einzelnen String sortieren. Ein einziger Aufruf von
\osuefunc{strcmp(3)} als Merge-Implementation gen{\"u}gt um zu entscheiden welche
Ausgabe der Kinder zuerst vom Elternprozess ausgegeben werden soll. Wenn
das funktioniert, können Sie Ihre Merge-Implementierung verallgemeinern, um auch
mehr als zwei Strings sortieren zu können.

Um Fehlermeldungen und Debug-Messages auszugeben, verwenden Sie stets
\osueglvar{stderr}, da die \osueglvar{stdout} in den meisten Fällen umgeleitet ist.

Zum Testen der Merge-Implementierung können Sie mit \osuefunc{execlp(3)} statt
Forksort auch \osuecmd{sort(1)} aufrufen. Für die endgültige Abgabe ist diese
Vorgehensweise nicht gültig, zum Testen aber durchaus sinnvoll.

\osueguidelinestwo

\end{document}
